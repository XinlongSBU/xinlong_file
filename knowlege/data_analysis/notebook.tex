
% Default to the notebook output style

    


% Inherit from the specified cell style.




    
\documentclass[11pt]{article}

    
    
    \usepackage[T1]{fontenc}
    % Nicer default font (+ math font) than Computer Modern for most use cases
    \usepackage{mathpazo}

    % Basic figure setup, for now with no caption control since it's done
    % automatically by Pandoc (which extracts ![](path) syntax from Markdown).
    \usepackage{graphicx}
    % We will generate all images so they have a width \maxwidth. This means
    % that they will get their normal width if they fit onto the page, but
    % are scaled down if they would overflow the margins.
    \makeatletter
    \def\maxwidth{\ifdim\Gin@nat@width>\linewidth\linewidth
    \else\Gin@nat@width\fi}
    \makeatother
    \let\Oldincludegraphics\includegraphics
    % Set max figure width to be 80% of text width, for now hardcoded.
    \renewcommand{\includegraphics}[1]{\Oldincludegraphics[width=.8\maxwidth]{#1}}
    % Ensure that by default, figures have no caption (until we provide a
    % proper Figure object with a Caption API and a way to capture that
    % in the conversion process - todo).
    \usepackage{caption}
    \DeclareCaptionLabelFormat{nolabel}{}
    \captionsetup{labelformat=nolabel}

    \usepackage{adjustbox} % Used to constrain images to a maximum size 
    \usepackage{xcolor} % Allow colors to be defined
    \usepackage{enumerate} % Needed for markdown enumerations to work
    \usepackage{geometry} % Used to adjust the document margins
    \usepackage{amsmath} % Equations
    \usepackage{amssymb} % Equations
    \usepackage{textcomp} % defines textquotesingle
    % Hack from http://tex.stackexchange.com/a/47451/13684:
    \AtBeginDocument{%
        \def\PYZsq{\textquotesingle}% Upright quotes in Pygmentized code
    }
    \usepackage{upquote} % Upright quotes for verbatim code
    \usepackage{eurosym} % defines \euro
    \usepackage[mathletters]{ucs} % Extended unicode (utf-8) support
    \usepackage[utf8x]{inputenc} % Allow utf-8 characters in the tex document
    \usepackage{fancyvrb} % verbatim replacement that allows latex
    \usepackage{grffile} % extends the file name processing of package graphics 
                         % to support a larger range 
    % The hyperref package gives us a pdf with properly built
    % internal navigation ('pdf bookmarks' for the table of contents,
    % internal cross-reference links, web links for URLs, etc.)
    \usepackage{hyperref}
    \usepackage{longtable} % longtable support required by pandoc >1.10
    \usepackage{booktabs}  % table support for pandoc > 1.12.2
    \usepackage[inline]{enumitem} % IRkernel/repr support (it uses the enumerate* environment)
    \usepackage[normalem]{ulem} % ulem is needed to support strikethroughs (\sout)
                                % normalem makes italics be italics, not underlines
    

    
    
    % Colors for the hyperref package
    \definecolor{urlcolor}{rgb}{0,.145,.698}
    \definecolor{linkcolor}{rgb}{.71,0.21,0.01}
    \definecolor{citecolor}{rgb}{.12,.54,.11}

    % ANSI colors
    \definecolor{ansi-black}{HTML}{3E424D}
    \definecolor{ansi-black-intense}{HTML}{282C36}
    \definecolor{ansi-red}{HTML}{E75C58}
    \definecolor{ansi-red-intense}{HTML}{B22B31}
    \definecolor{ansi-green}{HTML}{00A250}
    \definecolor{ansi-green-intense}{HTML}{007427}
    \definecolor{ansi-yellow}{HTML}{DDB62B}
    \definecolor{ansi-yellow-intense}{HTML}{B27D12}
    \definecolor{ansi-blue}{HTML}{208FFB}
    \definecolor{ansi-blue-intense}{HTML}{0065CA}
    \definecolor{ansi-magenta}{HTML}{D160C4}
    \definecolor{ansi-magenta-intense}{HTML}{A03196}
    \definecolor{ansi-cyan}{HTML}{60C6C8}
    \definecolor{ansi-cyan-intense}{HTML}{258F8F}
    \definecolor{ansi-white}{HTML}{C5C1B4}
    \definecolor{ansi-white-intense}{HTML}{A1A6B2}

    % commands and environments needed by pandoc snippets
    % extracted from the output of `pandoc -s`
    \providecommand{\tightlist}{%
      \setlength{\itemsep}{0pt}\setlength{\parskip}{0pt}}
    \DefineVerbatimEnvironment{Highlighting}{Verbatim}{commandchars=\\\{\}}
    % Add ',fontsize=\small' for more characters per line
    \newenvironment{Shaded}{}{}
    \newcommand{\KeywordTok}[1]{\textcolor[rgb]{0.00,0.44,0.13}{\textbf{{#1}}}}
    \newcommand{\DataTypeTok}[1]{\textcolor[rgb]{0.56,0.13,0.00}{{#1}}}
    \newcommand{\DecValTok}[1]{\textcolor[rgb]{0.25,0.63,0.44}{{#1}}}
    \newcommand{\BaseNTok}[1]{\textcolor[rgb]{0.25,0.63,0.44}{{#1}}}
    \newcommand{\FloatTok}[1]{\textcolor[rgb]{0.25,0.63,0.44}{{#1}}}
    \newcommand{\CharTok}[1]{\textcolor[rgb]{0.25,0.44,0.63}{{#1}}}
    \newcommand{\StringTok}[1]{\textcolor[rgb]{0.25,0.44,0.63}{{#1}}}
    \newcommand{\CommentTok}[1]{\textcolor[rgb]{0.38,0.63,0.69}{\textit{{#1}}}}
    \newcommand{\OtherTok}[1]{\textcolor[rgb]{0.00,0.44,0.13}{{#1}}}
    \newcommand{\AlertTok}[1]{\textcolor[rgb]{1.00,0.00,0.00}{\textbf{{#1}}}}
    \newcommand{\FunctionTok}[1]{\textcolor[rgb]{0.02,0.16,0.49}{{#1}}}
    \newcommand{\RegionMarkerTok}[1]{{#1}}
    \newcommand{\ErrorTok}[1]{\textcolor[rgb]{1.00,0.00,0.00}{\textbf{{#1}}}}
    \newcommand{\NormalTok}[1]{{#1}}
    
    % Additional commands for more recent versions of Pandoc
    \newcommand{\ConstantTok}[1]{\textcolor[rgb]{0.53,0.00,0.00}{{#1}}}
    \newcommand{\SpecialCharTok}[1]{\textcolor[rgb]{0.25,0.44,0.63}{{#1}}}
    \newcommand{\VerbatimStringTok}[1]{\textcolor[rgb]{0.25,0.44,0.63}{{#1}}}
    \newcommand{\SpecialStringTok}[1]{\textcolor[rgb]{0.73,0.40,0.53}{{#1}}}
    \newcommand{\ImportTok}[1]{{#1}}
    \newcommand{\DocumentationTok}[1]{\textcolor[rgb]{0.73,0.13,0.13}{\textit{{#1}}}}
    \newcommand{\AnnotationTok}[1]{\textcolor[rgb]{0.38,0.63,0.69}{\textbf{\textit{{#1}}}}}
    \newcommand{\CommentVarTok}[1]{\textcolor[rgb]{0.38,0.63,0.69}{\textbf{\textit{{#1}}}}}
    \newcommand{\VariableTok}[1]{\textcolor[rgb]{0.10,0.09,0.49}{{#1}}}
    \newcommand{\ControlFlowTok}[1]{\textcolor[rgb]{0.00,0.44,0.13}{\textbf{{#1}}}}
    \newcommand{\OperatorTok}[1]{\textcolor[rgb]{0.40,0.40,0.40}{{#1}}}
    \newcommand{\BuiltInTok}[1]{{#1}}
    \newcommand{\ExtensionTok}[1]{{#1}}
    \newcommand{\PreprocessorTok}[1]{\textcolor[rgb]{0.74,0.48,0.00}{{#1}}}
    \newcommand{\AttributeTok}[1]{\textcolor[rgb]{0.49,0.56,0.16}{{#1}}}
    \newcommand{\InformationTok}[1]{\textcolor[rgb]{0.38,0.63,0.69}{\textbf{\textit{{#1}}}}}
    \newcommand{\WarningTok}[1]{\textcolor[rgb]{0.38,0.63,0.69}{\textbf{\textit{{#1}}}}}
    
    
    % Define a nice break command that doesn't care if a line doesn't already
    % exist.
    \def\br{\hspace*{\fill} \\* }
    % Math Jax compatability definitions
    \def\gt{>}
    \def\lt{<}
    % Document parameters
    \title{data\_cleaning}
    
    
    

    % Pygments definitions
    
\makeatletter
\def\PY@reset{\let\PY@it=\relax \let\PY@bf=\relax%
    \let\PY@ul=\relax \let\PY@tc=\relax%
    \let\PY@bc=\relax \let\PY@ff=\relax}
\def\PY@tok#1{\csname PY@tok@#1\endcsname}
\def\PY@toks#1+{\ifx\relax#1\empty\else%
    \PY@tok{#1}\expandafter\PY@toks\fi}
\def\PY@do#1{\PY@bc{\PY@tc{\PY@ul{%
    \PY@it{\PY@bf{\PY@ff{#1}}}}}}}
\def\PY#1#2{\PY@reset\PY@toks#1+\relax+\PY@do{#2}}

\expandafter\def\csname PY@tok@w\endcsname{\def\PY@tc##1{\textcolor[rgb]{0.73,0.73,0.73}{##1}}}
\expandafter\def\csname PY@tok@c\endcsname{\let\PY@it=\textit\def\PY@tc##1{\textcolor[rgb]{0.25,0.50,0.50}{##1}}}
\expandafter\def\csname PY@tok@cp\endcsname{\def\PY@tc##1{\textcolor[rgb]{0.74,0.48,0.00}{##1}}}
\expandafter\def\csname PY@tok@k\endcsname{\let\PY@bf=\textbf\def\PY@tc##1{\textcolor[rgb]{0.00,0.50,0.00}{##1}}}
\expandafter\def\csname PY@tok@kp\endcsname{\def\PY@tc##1{\textcolor[rgb]{0.00,0.50,0.00}{##1}}}
\expandafter\def\csname PY@tok@kt\endcsname{\def\PY@tc##1{\textcolor[rgb]{0.69,0.00,0.25}{##1}}}
\expandafter\def\csname PY@tok@o\endcsname{\def\PY@tc##1{\textcolor[rgb]{0.40,0.40,0.40}{##1}}}
\expandafter\def\csname PY@tok@ow\endcsname{\let\PY@bf=\textbf\def\PY@tc##1{\textcolor[rgb]{0.67,0.13,1.00}{##1}}}
\expandafter\def\csname PY@tok@nb\endcsname{\def\PY@tc##1{\textcolor[rgb]{0.00,0.50,0.00}{##1}}}
\expandafter\def\csname PY@tok@nf\endcsname{\def\PY@tc##1{\textcolor[rgb]{0.00,0.00,1.00}{##1}}}
\expandafter\def\csname PY@tok@nc\endcsname{\let\PY@bf=\textbf\def\PY@tc##1{\textcolor[rgb]{0.00,0.00,1.00}{##1}}}
\expandafter\def\csname PY@tok@nn\endcsname{\let\PY@bf=\textbf\def\PY@tc##1{\textcolor[rgb]{0.00,0.00,1.00}{##1}}}
\expandafter\def\csname PY@tok@ne\endcsname{\let\PY@bf=\textbf\def\PY@tc##1{\textcolor[rgb]{0.82,0.25,0.23}{##1}}}
\expandafter\def\csname PY@tok@nv\endcsname{\def\PY@tc##1{\textcolor[rgb]{0.10,0.09,0.49}{##1}}}
\expandafter\def\csname PY@tok@no\endcsname{\def\PY@tc##1{\textcolor[rgb]{0.53,0.00,0.00}{##1}}}
\expandafter\def\csname PY@tok@nl\endcsname{\def\PY@tc##1{\textcolor[rgb]{0.63,0.63,0.00}{##1}}}
\expandafter\def\csname PY@tok@ni\endcsname{\let\PY@bf=\textbf\def\PY@tc##1{\textcolor[rgb]{0.60,0.60,0.60}{##1}}}
\expandafter\def\csname PY@tok@na\endcsname{\def\PY@tc##1{\textcolor[rgb]{0.49,0.56,0.16}{##1}}}
\expandafter\def\csname PY@tok@nt\endcsname{\let\PY@bf=\textbf\def\PY@tc##1{\textcolor[rgb]{0.00,0.50,0.00}{##1}}}
\expandafter\def\csname PY@tok@nd\endcsname{\def\PY@tc##1{\textcolor[rgb]{0.67,0.13,1.00}{##1}}}
\expandafter\def\csname PY@tok@s\endcsname{\def\PY@tc##1{\textcolor[rgb]{0.73,0.13,0.13}{##1}}}
\expandafter\def\csname PY@tok@sd\endcsname{\let\PY@it=\textit\def\PY@tc##1{\textcolor[rgb]{0.73,0.13,0.13}{##1}}}
\expandafter\def\csname PY@tok@si\endcsname{\let\PY@bf=\textbf\def\PY@tc##1{\textcolor[rgb]{0.73,0.40,0.53}{##1}}}
\expandafter\def\csname PY@tok@se\endcsname{\let\PY@bf=\textbf\def\PY@tc##1{\textcolor[rgb]{0.73,0.40,0.13}{##1}}}
\expandafter\def\csname PY@tok@sr\endcsname{\def\PY@tc##1{\textcolor[rgb]{0.73,0.40,0.53}{##1}}}
\expandafter\def\csname PY@tok@ss\endcsname{\def\PY@tc##1{\textcolor[rgb]{0.10,0.09,0.49}{##1}}}
\expandafter\def\csname PY@tok@sx\endcsname{\def\PY@tc##1{\textcolor[rgb]{0.00,0.50,0.00}{##1}}}
\expandafter\def\csname PY@tok@m\endcsname{\def\PY@tc##1{\textcolor[rgb]{0.40,0.40,0.40}{##1}}}
\expandafter\def\csname PY@tok@gh\endcsname{\let\PY@bf=\textbf\def\PY@tc##1{\textcolor[rgb]{0.00,0.00,0.50}{##1}}}
\expandafter\def\csname PY@tok@gu\endcsname{\let\PY@bf=\textbf\def\PY@tc##1{\textcolor[rgb]{0.50,0.00,0.50}{##1}}}
\expandafter\def\csname PY@tok@gd\endcsname{\def\PY@tc##1{\textcolor[rgb]{0.63,0.00,0.00}{##1}}}
\expandafter\def\csname PY@tok@gi\endcsname{\def\PY@tc##1{\textcolor[rgb]{0.00,0.63,0.00}{##1}}}
\expandafter\def\csname PY@tok@gr\endcsname{\def\PY@tc##1{\textcolor[rgb]{1.00,0.00,0.00}{##1}}}
\expandafter\def\csname PY@tok@ge\endcsname{\let\PY@it=\textit}
\expandafter\def\csname PY@tok@gs\endcsname{\let\PY@bf=\textbf}
\expandafter\def\csname PY@tok@gp\endcsname{\let\PY@bf=\textbf\def\PY@tc##1{\textcolor[rgb]{0.00,0.00,0.50}{##1}}}
\expandafter\def\csname PY@tok@go\endcsname{\def\PY@tc##1{\textcolor[rgb]{0.53,0.53,0.53}{##1}}}
\expandafter\def\csname PY@tok@gt\endcsname{\def\PY@tc##1{\textcolor[rgb]{0.00,0.27,0.87}{##1}}}
\expandafter\def\csname PY@tok@err\endcsname{\def\PY@bc##1{\setlength{\fboxsep}{0pt}\fcolorbox[rgb]{1.00,0.00,0.00}{1,1,1}{\strut ##1}}}
\expandafter\def\csname PY@tok@kc\endcsname{\let\PY@bf=\textbf\def\PY@tc##1{\textcolor[rgb]{0.00,0.50,0.00}{##1}}}
\expandafter\def\csname PY@tok@kd\endcsname{\let\PY@bf=\textbf\def\PY@tc##1{\textcolor[rgb]{0.00,0.50,0.00}{##1}}}
\expandafter\def\csname PY@tok@kn\endcsname{\let\PY@bf=\textbf\def\PY@tc##1{\textcolor[rgb]{0.00,0.50,0.00}{##1}}}
\expandafter\def\csname PY@tok@kr\endcsname{\let\PY@bf=\textbf\def\PY@tc##1{\textcolor[rgb]{0.00,0.50,0.00}{##1}}}
\expandafter\def\csname PY@tok@bp\endcsname{\def\PY@tc##1{\textcolor[rgb]{0.00,0.50,0.00}{##1}}}
\expandafter\def\csname PY@tok@fm\endcsname{\def\PY@tc##1{\textcolor[rgb]{0.00,0.00,1.00}{##1}}}
\expandafter\def\csname PY@tok@vc\endcsname{\def\PY@tc##1{\textcolor[rgb]{0.10,0.09,0.49}{##1}}}
\expandafter\def\csname PY@tok@vg\endcsname{\def\PY@tc##1{\textcolor[rgb]{0.10,0.09,0.49}{##1}}}
\expandafter\def\csname PY@tok@vi\endcsname{\def\PY@tc##1{\textcolor[rgb]{0.10,0.09,0.49}{##1}}}
\expandafter\def\csname PY@tok@vm\endcsname{\def\PY@tc##1{\textcolor[rgb]{0.10,0.09,0.49}{##1}}}
\expandafter\def\csname PY@tok@sa\endcsname{\def\PY@tc##1{\textcolor[rgb]{0.73,0.13,0.13}{##1}}}
\expandafter\def\csname PY@tok@sb\endcsname{\def\PY@tc##1{\textcolor[rgb]{0.73,0.13,0.13}{##1}}}
\expandafter\def\csname PY@tok@sc\endcsname{\def\PY@tc##1{\textcolor[rgb]{0.73,0.13,0.13}{##1}}}
\expandafter\def\csname PY@tok@dl\endcsname{\def\PY@tc##1{\textcolor[rgb]{0.73,0.13,0.13}{##1}}}
\expandafter\def\csname PY@tok@s2\endcsname{\def\PY@tc##1{\textcolor[rgb]{0.73,0.13,0.13}{##1}}}
\expandafter\def\csname PY@tok@sh\endcsname{\def\PY@tc##1{\textcolor[rgb]{0.73,0.13,0.13}{##1}}}
\expandafter\def\csname PY@tok@s1\endcsname{\def\PY@tc##1{\textcolor[rgb]{0.73,0.13,0.13}{##1}}}
\expandafter\def\csname PY@tok@mb\endcsname{\def\PY@tc##1{\textcolor[rgb]{0.40,0.40,0.40}{##1}}}
\expandafter\def\csname PY@tok@mf\endcsname{\def\PY@tc##1{\textcolor[rgb]{0.40,0.40,0.40}{##1}}}
\expandafter\def\csname PY@tok@mh\endcsname{\def\PY@tc##1{\textcolor[rgb]{0.40,0.40,0.40}{##1}}}
\expandafter\def\csname PY@tok@mi\endcsname{\def\PY@tc##1{\textcolor[rgb]{0.40,0.40,0.40}{##1}}}
\expandafter\def\csname PY@tok@il\endcsname{\def\PY@tc##1{\textcolor[rgb]{0.40,0.40,0.40}{##1}}}
\expandafter\def\csname PY@tok@mo\endcsname{\def\PY@tc##1{\textcolor[rgb]{0.40,0.40,0.40}{##1}}}
\expandafter\def\csname PY@tok@ch\endcsname{\let\PY@it=\textit\def\PY@tc##1{\textcolor[rgb]{0.25,0.50,0.50}{##1}}}
\expandafter\def\csname PY@tok@cm\endcsname{\let\PY@it=\textit\def\PY@tc##1{\textcolor[rgb]{0.25,0.50,0.50}{##1}}}
\expandafter\def\csname PY@tok@cpf\endcsname{\let\PY@it=\textit\def\PY@tc##1{\textcolor[rgb]{0.25,0.50,0.50}{##1}}}
\expandafter\def\csname PY@tok@c1\endcsname{\let\PY@it=\textit\def\PY@tc##1{\textcolor[rgb]{0.25,0.50,0.50}{##1}}}
\expandafter\def\csname PY@tok@cs\endcsname{\let\PY@it=\textit\def\PY@tc##1{\textcolor[rgb]{0.25,0.50,0.50}{##1}}}

\def\PYZbs{\char`\\}
\def\PYZus{\char`\_}
\def\PYZob{\char`\{}
\def\PYZcb{\char`\}}
\def\PYZca{\char`\^}
\def\PYZam{\char`\&}
\def\PYZlt{\char`\<}
\def\PYZgt{\char`\>}
\def\PYZsh{\char`\#}
\def\PYZpc{\char`\%}
\def\PYZdl{\char`\$}
\def\PYZhy{\char`\-}
\def\PYZsq{\char`\'}
\def\PYZdq{\char`\"}
\def\PYZti{\char`\~}
% for compatibility with earlier versions
\def\PYZat{@}
\def\PYZlb{[}
\def\PYZrb{]}
\makeatother


    % Exact colors from NB
    \definecolor{incolor}{rgb}{0.0, 0.0, 0.5}
    \definecolor{outcolor}{rgb}{0.545, 0.0, 0.0}



    
    % Prevent overflowing lines due to hard-to-break entities
    \sloppy 
    % Setup hyperref package
    \hypersetup{
      breaklinks=true,  % so long urls are correctly broken across lines
      colorlinks=true,
      urlcolor=urlcolor,
      linkcolor=linkcolor,
      citecolor=citecolor,
      }
    % Slightly bigger margins than the latex defaults
    
    \geometry{verbose,tmargin=1in,bmargin=1in,lmargin=1in,rmargin=1in}
    
    

    \begin{document}
    
    
    \maketitle
    
    

    
    \section{1 Problem description}\label{problem-description}

    The aim of this data exercise is to test a data source, called 'Signal',
which claims to be predictive of future returns of the SP500 index (use
SPY as a proxy).

There will be two main parts in the following text. - Data cleaning:
identify any errors in the data - Time series analysis: to check if
'Signal' could be used to predict future values of 'ClosePrice' of
SP500.

    \begin{Verbatim}[commandchars=\\\{\}]
{\color{incolor}In [{\color{incolor}1}]:} \PY{k+kn}{import} \PY{n+nn}{pandas} \PY{k}{as} \PY{n+nn}{pd}
        \PY{k+kn}{import} \PY{n+nn}{pandas\PYZus{}market\PYZus{}calendars} \PY{k}{as} \PY{n+nn}{mcal}
        \PY{k+kn}{import} \PY{n+nn}{numpy} \PY{k}{as} \PY{n+nn}{np}
        \PY{k+kn}{import} \PY{n+nn}{matplotlib}\PY{n+nn}{.}\PY{n+nn}{pyplot} \PY{k}{as} \PY{n+nn}{plt}
        \PY{k+kn}{import} \PY{n+nn}{matplotlib} \PY{k}{as} \PY{n+nn}{mpl}
        \PY{k+kn}{import} \PY{n+nn}{statsmodels}\PY{n+nn}{.}\PY{n+nn}{api} \PY{k}{as} \PY{n+nn}{sm}
        \PY{k+kn}{from} \PY{n+nn}{statsmodels}\PY{n+nn}{.}\PY{n+nn}{graphics}\PY{n+nn}{.}\PY{n+nn}{tsaplots} \PY{k}{import} \PY{n}{plot\PYZus{}acf}\PY{p}{,} \PY{n}{plot\PYZus{}pacf}
        \PY{k+kn}{from} \PY{n+nn}{statsmodels}\PY{n+nn}{.}\PY{n+nn}{tsa}\PY{n+nn}{.}\PY{n+nn}{stattools} \PY{k}{import} \PY{n}{adfuller}\PY{p}{,} \PY{n}{kpss}
        \PY{k+kn}{from} \PY{n+nn}{statsmodels}\PY{n+nn}{.}\PY{n+nn}{tsa}\PY{n+nn}{.}\PY{n+nn}{seasonal} \PY{k}{import} \PY{n}{seasonal\PYZus{}decompose}
        \PY{k+kn}{from} \PY{n+nn}{statsmodels}\PY{n+nn}{.}\PY{n+nn}{tsa}\PY{n+nn}{.}\PY{n+nn}{arima\PYZus{}model} \PY{k}{import} \PY{n}{ARIMA}
        \PY{k+kn}{import} \PY{n+nn}{pmdarima} \PY{k}{as} \PY{n+nn}{pm}
\end{Verbatim}


    Import data from the excel file:

    \begin{Verbatim}[commandchars=\\\{\}]
{\color{incolor}In [{\color{incolor}2}]:} \PY{n}{df} \PY{o}{=} \PY{n}{pd}\PY{o}{.}\PY{n}{read\PYZus{}excel}\PY{p}{(}\PY{l+s+s2}{\PYZdq{}}\PY{l+s+s2}{ResearchDatasetV2.0.xlsx}\PY{l+s+s2}{\PYZdq{}}\PY{p}{)}\PY{c+c1}{\PYZsh{},index\PYZus{}col=\PYZsq{}Date\PYZsq{})}
        \PY{n+nb}{print}\PY{p}{(}\PY{n}{df}\PY{o}{.}\PY{n}{head} \PY{p}{(}\PY{p}{)}\PY{p}{)}
\end{Verbatim}


    \begin{Verbatim}[commandchars=\\\{\}]
       Date    Signal  ClosePrice
0  20120103  3.107767     127.495
1  20120104  3.107282     127.700
2  20120105  3.099757     128.040
3  20120106  3.134223     127.710
4  20120109  3.135922     128.020

    \end{Verbatim}

    \begin{Verbatim}[commandchars=\\\{\}]
{\color{incolor}In [{\color{incolor}3}]:} \PY{n+nb}{print}\PY{p}{(}\PY{n}{df}\PY{o}{.}\PY{n}{info}\PY{p}{(}\PY{p}{)}\PY{p}{)}
\end{Verbatim}


    \begin{Verbatim}[commandchars=\\\{\}]
<class 'pandas.core.frame.DataFrame'>
RangeIndex: 667 entries, 0 to 666
Data columns (total 3 columns):
Date          667 non-null int64
Signal        667 non-null float64
ClosePrice    667 non-null float64
dtypes: float64(2), int64(1)
memory usage: 15.7 KB
None

    \end{Verbatim}

    There are three columns in the dataframe: - Date - Signal, which may be
predictive of future returns of the SP500 index (use SPY as a proxy) -
Close price, which is the SPY price

    \section{1 Data Cleaning}\label{data-cleaning}

    Firstly, Let's check if there are missing values.

    \begin{Verbatim}[commandchars=\\\{\}]
{\color{incolor}In [{\color{incolor}4}]:} \PY{n+nb}{print}\PY{p}{(}\PY{n}{df}\PY{o}{.}\PY{n}{isna}\PY{p}{(}\PY{p}{)}\PY{o}{.}\PY{n}{any}\PY{p}{(}\PY{p}{)}\PY{p}{)}
        \PY{n+nb}{print}\PY{p}{(}\PY{n}{df}\PY{o}{.}\PY{n}{isnull}\PY{p}{(}\PY{p}{)}\PY{o}{.}\PY{n}{any}\PY{p}{(}\PY{p}{)}\PY{p}{)}
\end{Verbatim}


    \begin{Verbatim}[commandchars=\\\{\}]
Date          False
Signal        False
ClosePrice    False
dtype: bool
Date          False
Signal        False
ClosePrice    False
dtype: bool

    \end{Verbatim}

    There is no missing value. let's convert type of the Date column from
int64 to datetime format

    \begin{Verbatim}[commandchars=\\\{\}]
{\color{incolor}In [{\color{incolor}5}]:} \PY{n}{df}\PY{p}{[}\PY{l+s+s1}{\PYZsq{}}\PY{l+s+s1}{Date}\PY{l+s+s1}{\PYZsq{}}\PY{p}{]} \PY{o}{=} \PY{n}{pd}\PY{o}{.}\PY{n}{to\PYZus{}datetime}\PY{p}{(}\PY{n}{df}\PY{p}{[}\PY{l+s+s1}{\PYZsq{}}\PY{l+s+s1}{Date}\PY{l+s+s1}{\PYZsq{}}\PY{p}{]}\PY{p}{,}\PY{n+nb}{format}\PY{o}{=}\PY{l+s+s1}{\PYZsq{}}\PY{l+s+s1}{\PYZpc{}}\PY{l+s+s1}{Y}\PY{l+s+s1}{\PYZpc{}}\PY{l+s+s1}{m}\PY{l+s+si}{\PYZpc{}d}\PY{l+s+s1}{\PYZsq{}}\PY{p}{,}\PY{n}{utc}\PY{o}{=}\PY{k+kc}{True}\PY{p}{)}
\end{Verbatim}


    Then we add a column "Day of Week" to the dataframe

    \begin{Verbatim}[commandchars=\\\{\}]
{\color{incolor}In [{\color{incolor}6}]:} \PY{n}{df}\PY{p}{[}\PY{l+s+s1}{\PYZsq{}}\PY{l+s+s1}{Day of Week}\PY{l+s+s1}{\PYZsq{}}\PY{p}{]} \PY{o}{=} \PY{n}{df}\PY{p}{[}\PY{l+s+s1}{\PYZsq{}}\PY{l+s+s1}{Date}\PY{l+s+s1}{\PYZsq{}}\PY{p}{]}\PY{o}{.}\PY{n}{dt}\PY{o}{.}\PY{n}{weekday\PYZus{}name}
\end{Verbatim}


    Let's check if the dates are unique

    \begin{Verbatim}[commandchars=\\\{\}]
{\color{incolor}In [{\color{incolor}7}]:} \PY{n+nb}{print}\PY{p}{(}\PY{n}{df}\PY{p}{[}\PY{l+s+s1}{\PYZsq{}}\PY{l+s+s1}{Date}\PY{l+s+s1}{\PYZsq{}}\PY{p}{]}\PY{o}{.}\PY{n}{is\PYZus{}unique}\PY{p}{)}
\end{Verbatim}


    \begin{Verbatim}[commandchars=\\\{\}]
True

    \end{Verbatim}

    We get the calendar of trading days for the date range of dataframe df

    \begin{Verbatim}[commandchars=\\\{\}]
{\color{incolor}In [{\color{incolor}8}]:} \PY{n}{nyse} \PY{o}{=} \PY{n}{mcal}\PY{o}{.}\PY{n}{get\PYZus{}calendar}\PY{p}{(}\PY{l+s+s1}{\PYZsq{}}\PY{l+s+s1}{NYSE}\PY{l+s+s1}{\PYZsq{}}\PY{p}{)}
        \PY{n}{val\PYZus{}day} \PY{o}{=} \PY{n}{nyse}\PY{o}{.}\PY{n}{valid\PYZus{}days}\PY{p}{(}\PY{n}{start\PYZus{}date}\PY{o}{=}\PY{n+nb}{min}\PY{p}{(}\PY{n}{df}\PY{o}{.}\PY{n}{Date}\PY{p}{)}\PY{p}{,} \PY{n}{end\PYZus{}date}\PY{o}{=}\PY{n+nb}{max}\PY{p}{(}\PY{n}{df}\PY{o}{.}\PY{n}{Date}\PY{p}{)}\PY{p}{)}
\end{Verbatim}


    Let's check if all the dates are trading days

    \begin{Verbatim}[commandchars=\\\{\}]
{\color{incolor}In [{\color{incolor}9}]:} \PY{n}{wrong\PYZus{}date}\PY{o}{=} \PY{n}{df}\PY{p}{[}\PY{o}{\PYZti{}}\PY{n}{df}\PY{p}{[}\PY{l+s+s1}{\PYZsq{}}\PY{l+s+s1}{Date}\PY{l+s+s1}{\PYZsq{}}\PY{p}{]}\PY{o}{.}\PY{n}{isin}\PY{p}{(}\PY{n}{val\PYZus{}day}\PY{p}{)}\PY{p}{]}
        \PY{n+nb}{print}\PY{p}{(}\PY{n}{wrong\PYZus{}date}\PY{p}{)}
\end{Verbatim}


    \begin{Verbatim}[commandchars=\\\{\}]
                         Date    Signal  ClosePrice Day of Week
494 2013-12-25 00:00:00+00:00  4.439806      182.93   Wednesday
499 2014-01-01 00:00:00+00:00  4.454369      184.69   Wednesday
525 2014-02-08 00:00:00+00:00  4.466505      179.68    Saturday
526 2014-02-09 00:00:00+00:00  4.466505      179.68      Sunday

    \end{Verbatim}

    We can see there are 4 days which are illegal. 2013-12-25 is Christmas
Day, 2014-01-01 is New Year's Day. 2014-02-08 and 2014-02-09 are
weekends. So we dope them with their corresponding rows.

    \begin{Verbatim}[commandchars=\\\{\}]
{\color{incolor}In [{\color{incolor}10}]:} \PY{n}{df} \PY{o}{=} \PY{n}{df}\PY{p}{[}\PY{n}{df}\PY{p}{[}\PY{l+s+s1}{\PYZsq{}}\PY{l+s+s1}{Date}\PY{l+s+s1}{\PYZsq{}}\PY{p}{]}\PY{o}{.}\PY{n}{isin}\PY{p}{(}\PY{n}{val\PYZus{}day}\PY{p}{)}\PY{p}{]}
\end{Verbatim}


    Let's check if there are missing trading dates in the dataframe:

    \begin{Verbatim}[commandchars=\\\{\}]
{\color{incolor}In [{\color{incolor}11}]:} \PY{n}{days} \PY{o}{=} \PY{n}{val\PYZus{}day}\PY{p}{[}\PY{o}{\PYZti{}}\PY{n}{val\PYZus{}day}\PY{o}{.}\PY{n}{isin}\PY{p}{(}\PY{n}{df}\PY{p}{[}\PY{l+s+s1}{\PYZsq{}}\PY{l+s+s1}{Date}\PY{l+s+s1}{\PYZsq{}}\PY{p}{]}\PY{p}{)}\PY{p}{]}
         \PY{n+nb}{print}\PY{p}{(}\PY{l+s+s2}{\PYZdq{}}\PY{l+s+s2}{The }\PY{l+s+si}{\PYZpc{}d}\PY{l+s+s2}{ missing dates are:}\PY{l+s+s2}{\PYZdq{}}\PY{o}{\PYZpc{}}\PY{p}{(}\PY{n+nb}{len}\PY{p}{(}\PY{n}{val\PYZus{}day}\PY{p}{)}\PY{o}{\PYZhy{}}\PY{n}{df}\PY{o}{.}\PY{n}{shape}\PY{p}{[}\PY{l+m+mi}{0}\PY{p}{]}\PY{p}{)}\PY{p}{)}
         \PY{p}{[}\PY{n+nb}{print}\PY{p}{(}\PY{n}{item}\PY{p}{)} \PY{k}{for} \PY{n}{item} \PY{o+ow}{in} \PY{n}{days}\PY{o}{.}\PY{n}{strftime}\PY{p}{(}\PY{l+s+s1}{\PYZsq{}}\PY{l+s+s1}{\PYZpc{}}\PY{l+s+s1}{Y\PYZhy{}}\PY{l+s+s1}{\PYZpc{}}\PY{l+s+s1}{m\PYZhy{}}\PY{l+s+si}{\PYZpc{}d}\PY{l+s+s1}{\PYZsq{}}\PY{p}{)}\PY{p}{]}
\end{Verbatim}


    \begin{Verbatim}[commandchars=\\\{\}]
The 6 missing dates are:
2013-01-14
2013-01-15
2013-01-16
2013-01-17
2014-01-06
2014-02-11

    \end{Verbatim}

\begin{Verbatim}[commandchars=\\\{\}]
{\color{outcolor}Out[{\color{outcolor}11}]:} [None, None, None, None, None, None]
\end{Verbatim}
            
    We interpolate the missing values

    \begin{Verbatim}[commandchars=\\\{\}]
{\color{incolor}In [{\color{incolor}12}]:} \PY{n+nb}{print}\PY{p}{(}\PY{n}{df}\PY{o}{.}\PY{n}{set\PYZus{}index}\PY{p}{(}\PY{l+s+s1}{\PYZsq{}}\PY{l+s+s1}{Date}\PY{l+s+s1}{\PYZsq{}}\PY{p}{)}\PY{p}{[}\PY{l+s+s1}{\PYZsq{}}\PY{l+s+s1}{2013\PYZhy{}01\PYZhy{}11}\PY{l+s+s1}{\PYZsq{}}\PY{p}{:}\PY{l+s+s1}{\PYZsq{}}\PY{l+s+s1}{2013\PYZhy{}01\PYZhy{}20}\PY{l+s+s1}{\PYZsq{}}\PY{p}{]}\PY{p}{)}
         \PY{n}{df} \PY{o}{=} \PY{n}{df}\PY{o}{.}\PY{n}{set\PYZus{}index}\PY{p}{(}\PY{l+s+s1}{\PYZsq{}}\PY{l+s+s1}{Date}\PY{l+s+s1}{\PYZsq{}}\PY{p}{)}\PY{o}{.}\PY{n}{reindex}\PY{p}{(}\PY{n}{val\PYZus{}day}\PY{p}{)}\PY{o}{.}\PY{n}{interpolate}\PY{p}{(}\PY{p}{)}
         \PY{n+nb}{print}\PY{p}{(}\PY{n}{df}\PY{p}{[}\PY{l+s+s1}{\PYZsq{}}\PY{l+s+s1}{2013\PYZhy{}01\PYZhy{}11}\PY{l+s+s1}{\PYZsq{}}\PY{p}{:}\PY{l+s+s1}{\PYZsq{}}\PY{l+s+s1}{2013\PYZhy{}01\PYZhy{}20}\PY{l+s+s1}{\PYZsq{}}\PY{p}{]}\PY{p}{)}
\end{Verbatim}


    \begin{Verbatim}[commandchars=\\\{\}]
                             Signal  ClosePrice Day of Week
Date                                                       
2013-01-11 00:00:00+00:00  3.569660      147.07      Friday
2013-01-18 00:00:00+00:00  3.625485      148.33      Friday
                             Signal  ClosePrice Day of Week
2013-01-11 00:00:00+00:00  3.569660     147.070      Friday
2013-01-14 00:00:00+00:00  3.580825     147.322         NaN
2013-01-15 00:00:00+00:00  3.591990     147.574         NaN
2013-01-16 00:00:00+00:00  3.603155     147.826         NaN
2013-01-17 00:00:00+00:00  3.614320     148.078         NaN
2013-01-18 00:00:00+00:00  3.625485     148.330      Friday

    \end{Verbatim}

    The 4 NaNs above should be Monday, Tuesday, Wednesday and Thursday.

    Then Let's check if there are outliers in the Signal column

    \begin{Verbatim}[commandchars=\\\{\}]
{\color{incolor}In [{\color{incolor}13}]:} \PY{n}{df}\PY{o}{.}\PY{n}{boxplot}\PY{p}{(}\PY{l+s+s1}{\PYZsq{}}\PY{l+s+s1}{Signal}\PY{l+s+s1}{\PYZsq{}}\PY{p}{)}
         \PY{n}{df}\PY{o}{.}\PY{n}{plot}\PY{p}{(}\PY{n}{x}\PY{o}{=}\PY{n}{df}\PY{o}{.}\PY{n}{index}\PY{p}{,} \PY{n}{y}\PY{o}{=}\PY{l+s+s1}{\PYZsq{}}\PY{l+s+s1}{Signal}\PY{l+s+s1}{\PYZsq{}}\PY{p}{)}
         \PY{n}{plt}\PY{o}{.}\PY{n}{xlabel}\PY{p}{(}\PY{l+s+s1}{\PYZsq{}}\PY{l+s+s1}{Time}\PY{l+s+s1}{\PYZsq{}}\PY{p}{)}
         \PY{n}{plt}\PY{o}{.}\PY{n}{ylabel}\PY{p}{(}\PY{l+s+s1}{\PYZsq{}}\PY{l+s+s1}{Signal}\PY{l+s+s1}{\PYZsq{}}\PY{p}{)}
         \PY{n}{plt}\PY{o}{.}\PY{n}{grid}\PY{p}{(}\PY{p}{)}
\end{Verbatim}


    \begin{center}
    \adjustimage{max size={0.9\linewidth}{0.9\paperheight}}{output_28_0.png}
    \end{center}
    { \hspace*{\fill} \\}
    
    \begin{center}
    \adjustimage{max size={0.9\linewidth}{0.9\paperheight}}{output_28_1.png}
    \end{center}
    { \hspace*{\fill} \\}
    
    From the figures above, we can see there are 2 data points which are
significantly larger than others. They are:

    \begin{Verbatim}[commandchars=\\\{\}]
{\color{incolor}In [{\color{incolor}14}]:} \PY{n}{wrong\PYZus{}signal\PYZus{}l} \PY{o}{=} \PY{n}{df}\PY{p}{[}\PY{n}{df}\PY{p}{[}\PY{l+s+s1}{\PYZsq{}}\PY{l+s+s1}{Signal}\PY{l+s+s1}{\PYZsq{}}\PY{p}{]} \PY{o}{\PYZgt{}} \PY{l+m+mi}{400}\PY{p}{]}
         \PY{n+nb}{print}\PY{p}{(}\PY{n}{wrong\PYZus{}signal\PYZus{}l}\PY{p}{)}
\end{Verbatim}


    \begin{Verbatim}[commandchars=\\\{\}]
                               Signal  ClosePrice Day of Week
2013-11-05 00:00:00+00:00  429.514563      176.27     Tuesday
2013-11-06 00:00:00+00:00  432.961165      177.17   Wednesday

    \end{Verbatim}

    Let's replace them with interpolated values and check again:

    \begin{Verbatim}[commandchars=\\\{\}]
{\color{incolor}In [{\color{incolor}15}]:} \PY{n}{df}\PY{o}{.}\PY{n}{loc}\PY{p}{[}\PY{n}{df}\PY{p}{[}\PY{l+s+s1}{\PYZsq{}}\PY{l+s+s1}{Signal}\PY{l+s+s1}{\PYZsq{}}\PY{p}{]}\PY{o}{\PYZgt{}}\PY{l+m+mi}{400}\PY{p}{,}\PY{l+s+s1}{\PYZsq{}}\PY{l+s+s1}{Signal}\PY{l+s+s1}{\PYZsq{}}\PY{p}{]} \PY{o}{=} \PY{n}{np}\PY{o}{.}\PY{n}{nan}
         \PY{n}{df}\PY{p}{[}\PY{l+s+s1}{\PYZsq{}}\PY{l+s+s1}{Signal}\PY{l+s+s1}{\PYZsq{}}\PY{p}{]} \PY{o}{=} \PY{n}{df}\PY{p}{[}\PY{l+s+s1}{\PYZsq{}}\PY{l+s+s1}{Signal}\PY{l+s+s1}{\PYZsq{}}\PY{p}{]}\PY{o}{.}\PY{n}{interpolate}\PY{p}{(}\PY{p}{)}
         \PY{n}{df}\PY{o}{.}\PY{n}{boxplot}\PY{p}{(}\PY{l+s+s2}{\PYZdq{}}\PY{l+s+s2}{Signal}\PY{l+s+s2}{\PYZdq{}}\PY{p}{)}
         \PY{n}{df}\PY{o}{.}\PY{n}{plot}\PY{p}{(}\PY{n}{x}\PY{o}{=}\PY{n}{df}\PY{o}{.}\PY{n}{index}\PY{p}{,} \PY{n}{y}\PY{o}{=}\PY{l+s+s1}{\PYZsq{}}\PY{l+s+s1}{Signal}\PY{l+s+s1}{\PYZsq{}}\PY{p}{)}
         \PY{n}{plt}\PY{o}{.}\PY{n}{xlabel}\PY{p}{(}\PY{l+s+s1}{\PYZsq{}}\PY{l+s+s1}{Time}\PY{l+s+s1}{\PYZsq{}}\PY{p}{)}
         \PY{n}{plt}\PY{o}{.}\PY{n}{grid}\PY{p}{(}\PY{p}{)}
\end{Verbatim}


    \begin{center}
    \adjustimage{max size={0.9\linewidth}{0.9\paperheight}}{output_32_0.png}
    \end{center}
    { \hspace*{\fill} \\}
    
    \begin{center}
    \adjustimage{max size={0.9\linewidth}{0.9\paperheight}}{output_32_1.png}
    \end{center}
    { \hspace*{\fill} \\}
    
    Again, there are also a few data points which are significantly smaller
than others. They are:

    \begin{Verbatim}[commandchars=\\\{\}]
{\color{incolor}In [{\color{incolor}16}]:} \PY{n}{wrong\PYZus{}signal\PYZus{}s} \PY{o}{=} \PY{n}{df}\PY{p}{[}\PY{n}{df}\PY{p}{[}\PY{l+s+s1}{\PYZsq{}}\PY{l+s+s1}{Signal}\PY{l+s+s1}{\PYZsq{}}\PY{p}{]}\PY{o}{\PYZlt{}}\PY{l+m+mi}{2}\PY{p}{]}
         \PY{n+nb}{print}\PY{p}{(}\PY{n}{wrong\PYZus{}signal\PYZus{}s}\PY{p}{)}
\end{Verbatim}


    \begin{Verbatim}[commandchars=\\\{\}]
                             Signal  ClosePrice Day of Week
2013-03-26 00:00:00+00:00 -3.802670    156.1900     Tuesday
2014-04-14 00:00:00+00:00  0.004560    182.9401      Monday
2014-04-15 00:00:00+00:00  0.454976    184.2000     Tuesday
2014-04-16 00:00:00+00:00  0.455898    186.1250   Wednesday

    \end{Verbatim}

    Let's replace them with interpolated values and then we get the figure
of time series 'Signal'.

    \begin{Verbatim}[commandchars=\\\{\}]
{\color{incolor}In [{\color{incolor}17}]:} \PY{n}{df}\PY{o}{.}\PY{n}{loc}\PY{p}{[}\PY{n}{df}\PY{p}{[}\PY{l+s+s1}{\PYZsq{}}\PY{l+s+s1}{Signal}\PY{l+s+s1}{\PYZsq{}}\PY{p}{]}\PY{o}{\PYZlt{}}\PY{l+m+mi}{2}\PY{p}{,}\PY{l+s+s1}{\PYZsq{}}\PY{l+s+s1}{Signal}\PY{l+s+s1}{\PYZsq{}}\PY{p}{]} \PY{o}{=} \PY{n}{np}\PY{o}{.}\PY{n}{nan}
         \PY{n}{df}\PY{p}{[}\PY{l+s+s1}{\PYZsq{}}\PY{l+s+s1}{Signal}\PY{l+s+s1}{\PYZsq{}}\PY{p}{]} \PY{o}{=} \PY{n}{df}\PY{p}{[}\PY{l+s+s1}{\PYZsq{}}\PY{l+s+s1}{Signal}\PY{l+s+s1}{\PYZsq{}}\PY{p}{]}\PY{o}{.}\PY{n}{interpolate}\PY{p}{(}\PY{p}{)}
         \PY{n+nb}{print}\PY{p}{(}\PY{n}{df}\PY{p}{[}\PY{l+s+s1}{\PYZsq{}}\PY{l+s+s1}{Signal}\PY{l+s+s1}{\PYZsq{}}\PY{p}{]}\PY{o}{.}\PY{n}{describe}\PY{p}{(}\PY{p}{)}\PY{p}{)}
         \PY{n}{df}\PY{o}{.}\PY{n}{plot}\PY{p}{(}\PY{n}{x}\PY{o}{=}\PY{n}{df}\PY{o}{.}\PY{n}{index}\PY{p}{,} \PY{n}{y}\PY{o}{=}\PY{l+s+s1}{\PYZsq{}}\PY{l+s+s1}{Signal}\PY{l+s+s1}{\PYZsq{}}\PY{p}{)}
         \PY{n}{plt}\PY{o}{.}\PY{n}{xlabel}\PY{p}{(}\PY{l+s+s1}{\PYZsq{}}\PY{l+s+s1}{Time}\PY{l+s+s1}{\PYZsq{}}\PY{p}{)}
         \PY{n}{plt}\PY{o}{.}\PY{n}{ylabel}\PY{p}{(}\PY{l+s+s1}{\PYZsq{}}\PY{l+s+s1}{Signal}\PY{l+s+s1}{\PYZsq{}}\PY{p}{)}
         \PY{n}{plt}\PY{o}{.}\PY{n}{grid}\PY{p}{(}\PY{p}{)}
\end{Verbatim}


    \begin{Verbatim}[commandchars=\\\{\}]
count    669.000000
mean       3.913538
std        0.523309
min        3.099757
25\%        3.422816
50\%        3.893689
75\%        4.405583
max        4.881311
Name: Signal, dtype: float64

    \end{Verbatim}

    \begin{center}
    \adjustimage{max size={0.9\linewidth}{0.9\paperheight}}{output_36_1.png}
    \end{center}
    { \hspace*{\fill} \\}
    
    We can do the same thing for the ClosePrice column

    \begin{Verbatim}[commandchars=\\\{\}]
{\color{incolor}In [{\color{incolor}18}]:} \PY{n}{df}\PY{o}{.}\PY{n}{boxplot}\PY{p}{(}\PY{l+s+s2}{\PYZdq{}}\PY{l+s+s2}{ClosePrice}\PY{l+s+s2}{\PYZdq{}}\PY{p}{)}
         \PY{n}{df}\PY{o}{.}\PY{n}{plot}\PY{p}{(}\PY{n}{x}\PY{o}{=}\PY{n}{df}\PY{o}{.}\PY{n}{index}\PY{p}{,} \PY{n}{y}\PY{o}{=}\PY{l+s+s1}{\PYZsq{}}\PY{l+s+s1}{ClosePrice}\PY{l+s+s1}{\PYZsq{}}\PY{p}{)}
         \PY{n}{plt}\PY{o}{.}\PY{n}{xlabel}\PY{p}{(}\PY{l+s+s1}{\PYZsq{}}\PY{l+s+s1}{Time}\PY{l+s+s1}{\PYZsq{}}\PY{p}{)}
         \PY{n}{plt}\PY{o}{.}\PY{n}{ylabel}\PY{p}{(}\PY{l+s+s1}{\PYZsq{}}\PY{l+s+s1}{ClosePrice}\PY{l+s+s1}{\PYZsq{}}\PY{p}{)}
\end{Verbatim}


\begin{Verbatim}[commandchars=\\\{\}]
{\color{outcolor}Out[{\color{outcolor}18}]:} Text(0, 0.5, 'ClosePrice')
\end{Verbatim}
            
    \begin{center}
    \adjustimage{max size={0.9\linewidth}{0.9\paperheight}}{output_38_1.png}
    \end{center}
    { \hspace*{\fill} \\}
    
    \begin{center}
    \adjustimage{max size={0.9\linewidth}{0.9\paperheight}}{output_38_2.png}
    \end{center}
    { \hspace*{\fill} \\}
    
    \begin{Verbatim}[commandchars=\\\{\}]
{\color{incolor}In [{\color{incolor}19}]:} \PY{n}{wrong\PYZus{}ClosePrice} \PY{o}{=} \PY{n}{df}\PY{p}{[}\PY{n}{df}\PY{p}{[}\PY{l+s+s1}{\PYZsq{}}\PY{l+s+s1}{ClosePrice}\PY{l+s+s1}{\PYZsq{}}\PY{p}{]}\PY{o}{\PYZgt{}}\PY{l+m+mi}{600}\PY{p}{]}
         \PY{n+nb}{print}\PY{p}{(}\PY{n}{wrong\PYZus{}ClosePrice}\PY{p}{)}
\end{Verbatim}


    \begin{Verbatim}[commandchars=\\\{\}]
                             Signal  ClosePrice Day of Week
2013-09-12 00:00:00+00:00  4.193204      618.95    Thursday
2013-09-13 00:00:00+00:00  4.143689      619.33      Friday
2013-09-16 00:00:00+00:00  4.124515      710.31      Monday

    \end{Verbatim}

    \begin{Verbatim}[commandchars=\\\{\}]
{\color{incolor}In [{\color{incolor}20}]:} \PY{n}{df}\PY{o}{.}\PY{n}{loc}\PY{p}{[}\PY{n}{df}\PY{p}{[}\PY{l+s+s1}{\PYZsq{}}\PY{l+s+s1}{ClosePrice}\PY{l+s+s1}{\PYZsq{}}\PY{p}{]}\PY{o}{\PYZgt{}}\PY{l+m+mi}{600}\PY{p}{,}\PY{l+s+s1}{\PYZsq{}}\PY{l+s+s1}{ClosePrice}\PY{l+s+s1}{\PYZsq{}}\PY{p}{]} \PY{o}{=} \PY{n}{np}\PY{o}{.}\PY{n}{nan}
         \PY{n}{df}\PY{p}{[}\PY{l+s+s1}{\PYZsq{}}\PY{l+s+s1}{ClosePrice}\PY{l+s+s1}{\PYZsq{}}\PY{p}{]} \PY{o}{=} \PY{n}{df}\PY{p}{[}\PY{l+s+s1}{\PYZsq{}}\PY{l+s+s1}{ClosePrice}\PY{l+s+s1}{\PYZsq{}}\PY{p}{]}\PY{o}{.}\PY{n}{interpolate}\PY{p}{(}\PY{p}{)}
         \PY{n+nb}{print}\PY{p}{(}\PY{n}{df}\PY{p}{[}\PY{l+s+s1}{\PYZsq{}}\PY{l+s+s1}{ClosePrice}\PY{l+s+s1}{\PYZsq{}}\PY{p}{]}\PY{o}{.}\PY{n}{describe}\PY{p}{(}\PY{p}{)}\PY{p}{)}
         \PY{n}{df}\PY{o}{.}\PY{n}{boxplot}\PY{p}{(}\PY{l+s+s2}{\PYZdq{}}\PY{l+s+s2}{ClosePrice}\PY{l+s+s2}{\PYZdq{}}\PY{p}{)}
         \PY{n}{df}\PY{o}{.}\PY{n}{plot}\PY{p}{(}\PY{n}{x}\PY{o}{=}\PY{n}{df}\PY{o}{.}\PY{n}{index}\PY{p}{,} \PY{n}{y}\PY{o}{=}\PY{l+s+s1}{\PYZsq{}}\PY{l+s+s1}{ClosePrice}\PY{l+s+s1}{\PYZsq{}}\PY{p}{)}
         \PY{n}{plt}\PY{o}{.}\PY{n}{xlabel}\PY{p}{(}\PY{l+s+s1}{\PYZsq{}}\PY{l+s+s1}{Time}\PY{l+s+s1}{\PYZsq{}}\PY{p}{)}
         \PY{n}{plt}\PY{o}{.}\PY{n}{ylabel}\PY{p}{(}\PY{l+s+s1}{\PYZsq{}}\PY{l+s+s1}{ClosePrice}\PY{l+s+s1}{\PYZsq{}}\PY{p}{)}
\end{Verbatim}


    \begin{Verbatim}[commandchars=\\\{\}]
count    669.000000
mean     160.874214
std       21.392170
min      127.495000
25\%      140.910000
50\%      159.300000
75\%      181.000000
max      200.710000
Name: ClosePrice, dtype: float64

    \end{Verbatim}

\begin{Verbatim}[commandchars=\\\{\}]
{\color{outcolor}Out[{\color{outcolor}20}]:} Text(0, 0.5, 'ClosePrice')
\end{Verbatim}
            
    \begin{center}
    \adjustimage{max size={0.9\linewidth}{0.9\paperheight}}{output_40_2.png}
    \end{center}
    { \hspace*{\fill} \\}
    
    \begin{center}
    \adjustimage{max size={0.9\linewidth}{0.9\paperheight}}{output_40_3.png}
    \end{center}
    { \hspace*{\fill} \\}
    
    Finally, We drop the column "Day of Week"

    \begin{Verbatim}[commandchars=\\\{\}]
{\color{incolor}In [{\color{incolor}21}]:} \PY{n}{df} \PY{o}{=} \PY{n}{df}\PY{o}{.}\PY{n}{drop}\PY{p}{(}\PY{n}{columns}\PY{o}{=}\PY{l+s+s1}{\PYZsq{}}\PY{l+s+s1}{Day of Week}\PY{l+s+s1}{\PYZsq{}}\PY{p}{)}
\end{Verbatim}


    Finally, Let's plot Signal and ClosePrice on top of each other

    \begin{Verbatim}[commandchars=\\\{\}]
{\color{incolor}In [{\color{incolor}22}]:} \PY{n}{plt}\PY{o}{.}\PY{n}{plot}\PY{p}{(}\PY{n}{df}\PY{p}{[}\PY{l+s+s1}{\PYZsq{}}\PY{l+s+s1}{Signal}\PY{l+s+s1}{\PYZsq{}}\PY{p}{]}\PY{o}{*}\PY{l+m+mf}{41.12}\PY{p}{,}\PY{n}{label}\PY{o}{=}\PY{l+s+s2}{\PYZdq{}}\PY{l+s+s2}{Signal*41.12}\PY{l+s+s2}{\PYZdq{}}\PY{p}{)}
         \PY{n}{plt}\PY{o}{.}\PY{n}{plot}\PY{p}{(}\PY{n}{df}\PY{p}{[}\PY{l+s+s1}{\PYZsq{}}\PY{l+s+s1}{ClosePrice}\PY{l+s+s1}{\PYZsq{}}\PY{p}{]}\PY{p}{,}\PY{n}{label}\PY{o}{=}\PY{l+s+s2}{\PYZdq{}}\PY{l+s+s2}{Closeprice}\PY{l+s+s2}{\PYZdq{}}\PY{p}{)}
         \PY{n}{plt}\PY{o}{.}\PY{n}{grid}\PY{p}{(}\PY{p}{)}
         \PY{n}{plt}\PY{o}{.}\PY{n}{show}\PY{p}{(}\PY{p}{)}
\end{Verbatim}


    \begin{center}
    \adjustimage{max size={0.9\linewidth}{0.9\paperheight}}{output_44_0.png}
    \end{center}
    { \hspace*{\fill} \\}
    
    Note that 'ClosePrice' is approximately 40 times bigger than 'Signal'.

    \section{2 Time series}\label{time-series}

    It seems that 'Signal' may be used to predict 'ClosePrice'. Let's first
have a glance on the decomposition plot on 'Signal'.

    \begin{Verbatim}[commandchars=\\\{\}]
{\color{incolor}In [{\color{incolor}23}]:} \PY{n}{decomposition} \PY{o}{=} \PY{n}{seasonal\PYZus{}decompose}\PY{p}{(}\PY{n}{df}\PY{p}{[}\PY{l+s+s1}{\PYZsq{}}\PY{l+s+s1}{Signal}\PY{l+s+s1}{\PYZsq{}}\PY{p}{]}\PY{p}{,}\PY{n}{freq}\PY{o}{=}\PY{l+m+mi}{22}\PY{p}{)}
         \PY{n}{fig} \PY{o}{=} \PY{n}{decomposition}\PY{o}{.}\PY{n}{plot}\PY{p}{(}\PY{p}{)}
         \PY{n}{plt}\PY{o}{.}\PY{n}{grid}\PY{p}{(}\PY{p}{)}
         \PY{n}{plt}\PY{o}{.}\PY{n}{show}\PY{p}{(}\PY{p}{)}
\end{Verbatim}


    \begin{Verbatim}[commandchars=\\\{\}]
/anaconda3/lib/python3.6/site-packages/ipykernel\_launcher.py:1: FutureWarning: the 'freq' keyword is deprecated, use 'period' instead
  """Entry point for launching an IPython kernel.

    \end{Verbatim}

    \begin{center}
    \adjustimage{max size={0.9\linewidth}{0.9\paperheight}}{output_48_1.png}
    \end{center}
    { \hspace*{\fill} \\}
    
    \subsubsection{2.1 Check Stationary}\label{check-stationary}

    From the figures above we can see that the time series is not
stationary. Let's confirm this using Augmented Dickey Fuller (ADF) Test

    \begin{Verbatim}[commandchars=\\\{\}]
{\color{incolor}In [{\color{incolor}24}]:} \PY{k}{def} \PY{n+nf}{adf\PYZus{}test}\PY{p}{(}\PY{n}{timeseries}\PY{p}{)}\PY{p}{:}
             \PY{c+c1}{\PYZsh{}Perform Dickey\PYZhy{}Fuller test:}
             \PY{n+nb}{print} \PY{p}{(}\PY{l+s+s1}{\PYZsq{}}\PY{l+s+s1}{Results of ADF Test:}\PY{l+s+s1}{\PYZsq{}}\PY{p}{)}
             \PY{n}{dftest} \PY{o}{=} \PY{n}{adfuller}\PY{p}{(}\PY{n}{timeseries}\PY{p}{,} \PY{n}{autolag}\PY{o}{=}\PY{l+s+s1}{\PYZsq{}}\PY{l+s+s1}{AIC}\PY{l+s+s1}{\PYZsq{}}\PY{p}{)}
             \PY{n}{dfoutput} \PY{o}{=} \PY{n}{pd}\PY{o}{.}\PY{n}{Series}\PY{p}{(}\PY{n}{dftest}\PY{p}{[}\PY{l+m+mi}{0}\PY{p}{:}\PY{l+m+mi}{4}\PY{p}{]}\PY{p}{,} \PY{n}{index}\PY{o}{=}\PY{p}{[}\PY{l+s+s1}{\PYZsq{}}\PY{l+s+s1}{Test Statistic}\PY{l+s+s1}{\PYZsq{}}\PY{p}{,}\PY{l+s+s1}{\PYZsq{}}\PY{l+s+s1}{p\PYZhy{}value}\PY{l+s+s1}{\PYZsq{}}\PY{p}{,}\PY{l+s+s1}{\PYZsq{}}\PY{l+s+s1}{\PYZsh{}Lags Used}\PY{l+s+s1}{\PYZsq{}}\PY{p}{,}\PY{l+s+s1}{\PYZsq{}}\PY{l+s+s1}{Number of Observations Used}\PY{l+s+s1}{\PYZsq{}}\PY{p}{]}\PY{p}{)}
             \PY{k}{for} \PY{n}{key}\PY{p}{,}\PY{n}{value} \PY{o+ow}{in} \PY{n}{dftest}\PY{p}{[}\PY{l+m+mi}{4}\PY{p}{]}\PY{o}{.}\PY{n}{items}\PY{p}{(}\PY{p}{)}\PY{p}{:}
                 \PY{n}{dfoutput}\PY{p}{[}\PY{l+s+s1}{\PYZsq{}}\PY{l+s+s1}{Critical Value (}\PY{l+s+si}{\PYZpc{}s}\PY{l+s+s1}{)}\PY{l+s+s1}{\PYZsq{}}\PY{o}{\PYZpc{}}\PY{k}{key}] = value
             \PY{n+nb}{print}\PY{p}{(}\PY{n}{dfoutput}\PY{p}{)}
         
         \PY{c+c1}{\PYZsh{}apply adf test on the series}
         \PY{n}{adf\PYZus{}test}\PY{p}{(}\PY{n}{df}\PY{p}{[}\PY{l+s+s1}{\PYZsq{}}\PY{l+s+s1}{Signal}\PY{l+s+s1}{\PYZsq{}}\PY{p}{]}\PY{p}{)}
\end{Verbatim}


    \begin{Verbatim}[commandchars=\\\{\}]
Results of ADF Test:
Test Statistic                  -0.306330
p-value                          0.924603
\#Lags Used                       2.000000
Number of Observations Used    666.000000
Critical Value (1\%)             -3.440207
Critical Value (5\%)             -2.865889
Critical Value (10\%)            -2.569086
dtype: float64

    \end{Verbatim}

    The test statistic is bigger than the critical values, which implies
that the time series 'Signal' is not stationary.

    Let's try Kwiatkowski-Phillips-Schmidt-Shin (KPSS) test to check it
again

    \begin{Verbatim}[commandchars=\\\{\}]
{\color{incolor}In [{\color{incolor}25}]:} \PY{k}{def} \PY{n+nf}{kpss\PYZus{}test}\PY{p}{(}\PY{n}{timeseries}\PY{p}{)}\PY{p}{:}
             \PY{n+nb}{print} \PY{p}{(}\PY{l+s+s1}{\PYZsq{}}\PY{l+s+s1}{Results of KPSS Test:}\PY{l+s+s1}{\PYZsq{}}\PY{p}{)}
             \PY{n}{kpsstest} \PY{o}{=} \PY{n}{kpss}\PY{p}{(}\PY{n}{timeseries}\PY{p}{,} \PY{n}{regression}\PY{o}{=}\PY{l+s+s1}{\PYZsq{}}\PY{l+s+s1}{c}\PY{l+s+s1}{\PYZsq{}}\PY{p}{)}
             \PY{n}{kpss\PYZus{}output} \PY{o}{=} \PY{n}{pd}\PY{o}{.}\PY{n}{Series}\PY{p}{(}\PY{n}{kpsstest}\PY{p}{[}\PY{l+m+mi}{0}\PY{p}{:}\PY{l+m+mi}{3}\PY{p}{]}\PY{p}{,} \PY{n}{index}\PY{o}{=}\PY{p}{[}\PY{l+s+s1}{\PYZsq{}}\PY{l+s+s1}{Test Statistic}\PY{l+s+s1}{\PYZsq{}}\PY{p}{,}\PY{l+s+s1}{\PYZsq{}}\PY{l+s+s1}{p\PYZhy{}value}\PY{l+s+s1}{\PYZsq{}}\PY{p}{,}\PY{l+s+s1}{\PYZsq{}}\PY{l+s+s1}{Lags Used}\PY{l+s+s1}{\PYZsq{}}\PY{p}{]}\PY{p}{)}
             \PY{k}{for} \PY{n}{key}\PY{p}{,}\PY{n}{value} \PY{o+ow}{in} \PY{n}{kpsstest}\PY{p}{[}\PY{l+m+mi}{3}\PY{p}{]}\PY{o}{.}\PY{n}{items}\PY{p}{(}\PY{p}{)}\PY{p}{:}
                 \PY{n}{kpss\PYZus{}output}\PY{p}{[}\PY{l+s+s1}{\PYZsq{}}\PY{l+s+s1}{Critical Value (}\PY{l+s+si}{\PYZpc{}s}\PY{l+s+s1}{)}\PY{l+s+s1}{\PYZsq{}}\PY{o}{\PYZpc{}}\PY{k}{key}] = value
             \PY{n+nb}{print}\PY{p}{(}\PY{n}{kpss\PYZus{}output}\PY{p}{)}
         \PY{n}{kpss\PYZus{}test}\PY{p}{(}\PY{n}{df}\PY{p}{[}\PY{l+s+s1}{\PYZsq{}}\PY{l+s+s1}{Signal}\PY{l+s+s1}{\PYZsq{}}\PY{p}{]}\PY{p}{)}
\end{Verbatim}


    \begin{Verbatim}[commandchars=\\\{\}]
Results of KPSS Test:
Test Statistic            3.241585
p-value                   0.010000
Lags Used                20.000000
Critical Value (10\%)      0.347000
Critical Value (5\%)       0.463000
Critical Value (2.5\%)     0.574000
Critical Value (1\%)       0.739000
dtype: float64

    \end{Verbatim}

    \begin{Verbatim}[commandchars=\\\{\}]
/anaconda3/lib/python3.6/site-packages/statsmodels/tsa/stattools.py:1661: FutureWarning: The behavior of using lags=None will change in the next release. Currently lags=None is the same as lags='legacy', and so a sample-size lag length is used. After the next release, the default will change to be the same as lags='auto' which uses an automatic lag length selection method. To silence this warning, either use 'auto' or 'legacy'
  warn(msg, FutureWarning)
/anaconda3/lib/python3.6/site-packages/statsmodels/tsa/stattools.py:1685: InterpolationWarning: p-value is smaller than the indicated p-value
  warn("p-value is smaller than the indicated p-value", InterpolationWarning)

    \end{Verbatim}

    With the test statistic bigger than the critical values, we reject the
null hypothesis, which confirms again that the time series 'Signal' is
not stationary. The p-value is less than 0.05.

    Next, let's do a quick differencing on the column 'Signal' to make the
time series stationary.

    \begin{Verbatim}[commandchars=\\\{\}]
{\color{incolor}In [{\color{incolor}26}]:} \PY{n}{df}\PY{p}{[}\PY{l+s+s1}{\PYZsq{}}\PY{l+s+s1}{Signal\PYZus{}diff}\PY{l+s+s1}{\PYZsq{}}\PY{p}{]} \PY{o}{=} \PY{n}{df}\PY{p}{[}\PY{l+s+s1}{\PYZsq{}}\PY{l+s+s1}{Signal}\PY{l+s+s1}{\PYZsq{}}\PY{p}{]} \PY{o}{\PYZhy{}} \PY{n}{df}\PY{p}{[}\PY{l+s+s1}{\PYZsq{}}\PY{l+s+s1}{Signal}\PY{l+s+s1}{\PYZsq{}}\PY{p}{]}\PY{o}{.}\PY{n}{shift}\PY{p}{(}\PY{l+m+mi}{1}\PY{p}{)}
         \PY{n}{df}\PY{p}{[}\PY{l+s+s1}{\PYZsq{}}\PY{l+s+s1}{Signal\PYZus{}diff}\PY{l+s+s1}{\PYZsq{}}\PY{p}{]}\PY{o}{.}\PY{n}{dropna}\PY{p}{(}\PY{p}{)}\PY{o}{.}\PY{n}{plot}\PY{p}{(}\PY{p}{)}
         \PY{n}{adf\PYZus{}test}\PY{p}{(}\PY{n}{df}\PY{p}{[}\PY{l+s+s1}{\PYZsq{}}\PY{l+s+s1}{Signal\PYZus{}diff}\PY{l+s+s1}{\PYZsq{}}\PY{p}{]}\PY{o}{.}\PY{n}{dropna}\PY{p}{(}\PY{p}{)}\PY{p}{)}
         \PY{n}{kpss\PYZus{}test}\PY{p}{(}\PY{n}{df}\PY{p}{[}\PY{l+s+s1}{\PYZsq{}}\PY{l+s+s1}{Signal\PYZus{}diff}\PY{l+s+s1}{\PYZsq{}}\PY{p}{]}\PY{o}{.}\PY{n}{dropna}\PY{p}{(}\PY{p}{)}\PY{p}{)}
\end{Verbatim}


    \begin{Verbatim}[commandchars=\\\{\}]
Results of ADF Test:

    \end{Verbatim}

    \begin{Verbatim}[commandchars=\\\{\}]
/anaconda3/lib/python3.6/site-packages/statsmodels/tsa/stattools.py:1661: FutureWarning: The behavior of using lags=None will change in the next release. Currently lags=None is the same as lags='legacy', and so a sample-size lag length is used. After the next release, the default will change to be the same as lags='auto' which uses an automatic lag length selection method. To silence this warning, either use 'auto' or 'legacy'
  warn(msg, FutureWarning)
/anaconda3/lib/python3.6/site-packages/statsmodels/tsa/stattools.py:1687: InterpolationWarning: p-value is greater than the indicated p-value
  warn("p-value is greater than the indicated p-value", InterpolationWarning)

    \end{Verbatim}

    \begin{Verbatim}[commandchars=\\\{\}]
Test Statistic                 -20.202023
p-value                          0.000000
\#Lags Used                       1.000000
Number of Observations Used    666.000000
Critical Value (1\%)             -3.440207
Critical Value (5\%)             -2.865889
Critical Value (10\%)            -2.569086
dtype: float64
Results of KPSS Test:
Test Statistic            0.050405
p-value                   0.100000
Lags Used                20.000000
Critical Value (10\%)      0.347000
Critical Value (5\%)       0.463000
Critical Value (2.5\%)     0.574000
Critical Value (1\%)       0.739000
dtype: float64

    \end{Verbatim}

    \begin{center}
    \adjustimage{max size={0.9\linewidth}{0.9\paperheight}}{output_57_3.png}
    \end{center}
    { \hspace*{\fill} \\}
    
    Both ADF test and KPSS test imply that the time series
'Signal\_log\_diff' is stationary.

    \subsubsection{2.2 ARIMA Model}\label{arima-model}

    Firstly, let's have a look at the Auto Correlation Function (ACF) figure
and Partial Auto Correlation Function (PACF) figure of 'Signal\_diff'.

    \begin{Verbatim}[commandchars=\\\{\}]
{\color{incolor}In [{\color{incolor}27}]:} \PY{n}{plot\PYZus{}acf}\PY{p}{(}\PY{n}{df}\PY{p}{[}\PY{l+s+s1}{\PYZsq{}}\PY{l+s+s1}{Signal\PYZus{}diff}\PY{l+s+s1}{\PYZsq{}}\PY{p}{]}\PY{o}{.}\PY{n}{dropna}\PY{p}{(}\PY{p}{)}\PY{p}{,} \PY{n}{lags}\PY{o}{=}\PY{l+m+mi}{20}\PY{p}{)}
         \PY{n}{plt}\PY{o}{.}\PY{n}{grid}\PY{p}{(}\PY{p}{)}
         \PY{n}{plt}\PY{o}{.}\PY{n}{title}\PY{p}{(}\PY{l+s+s1}{\PYZsq{}}\PY{l+s+s1}{Autocorrelation of Signal\PYZus{}diff}\PY{l+s+s1}{\PYZsq{}}\PY{p}{)}
         \PY{n}{plt}\PY{o}{.}\PY{n}{show}\PY{p}{(}\PY{p}{)}
         \PY{n}{plot\PYZus{}pacf}\PY{p}{(}\PY{n}{df}\PY{p}{[}\PY{l+s+s1}{\PYZsq{}}\PY{l+s+s1}{Signal\PYZus{}diff}\PY{l+s+s1}{\PYZsq{}}\PY{p}{]}\PY{o}{.}\PY{n}{dropna}\PY{p}{(}\PY{p}{)}\PY{p}{,} \PY{n}{lags}\PY{o}{=}\PY{l+m+mi}{20}\PY{p}{)}
         \PY{n}{plt}\PY{o}{.}\PY{n}{grid}\PY{p}{(}\PY{p}{)}
         \PY{n}{plt}\PY{o}{.}\PY{n}{title}\PY{p}{(}\PY{l+s+s1}{\PYZsq{}}\PY{l+s+s1}{Partial Autocorrelation of Signal\PYZus{}diff}\PY{l+s+s1}{\PYZsq{}}\PY{p}{)}
         \PY{n}{plt}\PY{o}{.}\PY{n}{show}\PY{p}{(}\PY{p}{)}
\end{Verbatim}


    \begin{center}
    \adjustimage{max size={0.9\linewidth}{0.9\paperheight}}{output_61_0.png}
    \end{center}
    { \hspace*{\fill} \\}
    
    \begin{center}
    \adjustimage{max size={0.9\linewidth}{0.9\paperheight}}{output_61_1.png}
    \end{center}
    { \hspace*{\fill} \\}
    
    We can conclude that p = q = 2 is a good choice to fit 'Signal\_diff'
using ARIMA Model. We set d = 1 since we did a differencing on the
column 'Signal'. However, We find that p = q = 1 has lower AIC. So we
try (p,d,q) = (1,1,1)

    \begin{Verbatim}[commandchars=\\\{\}]
{\color{incolor}In [{\color{incolor}28}]:} \PY{k}{class} \PY{n+nc}{Arima\PYZus{}model}\PY{p}{:}
             \PY{k}{def} \PY{n+nf}{\PYZus{}\PYZus{}init\PYZus{}\PYZus{}}\PY{p}{(}\PY{n+nb+bp}{self}\PY{p}{,}\PY{n}{ts}\PY{p}{,}\PY{n}{order}\PY{p}{)}\PY{p}{:}
                 \PY{c+c1}{\PYZsh{} fit model}
                 \PY{n+nb+bp}{self}\PY{o}{.}\PY{n}{order} \PY{o}{=} \PY{n}{order}
                 \PY{n}{model} \PY{o}{=} \PY{n}{ARIMA}\PY{p}{(}\PY{n}{ts}\PY{p}{,} \PY{n}{order}\PY{o}{=}\PY{n}{order}\PY{p}{)}
                 \PY{l+s+sd}{\PYZsq{}\PYZsq{}\PYZsq{}}
         \PY{l+s+sd}{        self.model = pm.auto\PYZus{}arima(ts, start\PYZus{}p=1, start\PYZus{}q=1,}
         \PY{l+s+sd}{                                   test=\PYZsq{}adf\PYZsq{},       \PYZsh{} use adftest to find optimal \PYZsq{}d\PYZsq{}}
         \PY{l+s+sd}{                                   max\PYZus{}p=4, max\PYZus{}q=4, \PYZsh{} maximum p and q}
         \PY{l+s+sd}{                                   m=1,              \PYZsh{} frequency of series}
         \PY{l+s+sd}{                                   d=None,           \PYZsh{} let model determine \PYZsq{}d\PYZsq{}}
         \PY{l+s+sd}{                                   seasonal=False,   \PYZsh{} No Seasonality}
         \PY{l+s+sd}{                                   start\PYZus{}P=0, }
         \PY{l+s+sd}{                                   D=0, }
         \PY{l+s+sd}{                                   trace=True,}
         \PY{l+s+sd}{                                   error\PYZus{}action=\PYZsq{}ignore\PYZsq{},  }
         \PY{l+s+sd}{                                   suppress\PYZus{}warnings=True, }
         \PY{l+s+sd}{                                   stepwise=True)}
         \PY{l+s+sd}{        \PYZsq{}\PYZsq{}\PYZsq{}}
                 \PY{n+nb+bp}{self}\PY{o}{.}\PY{n}{model\PYZus{}fit} \PY{o}{=} \PY{n}{model}\PY{o}{.}\PY{n}{fit}\PY{p}{(}\PY{n}{disp}\PY{o}{=}\PY{l+m+mi}{0}\PY{p}{)}
                 \PY{n+nb}{print}\PY{p}{(}\PY{n+nb+bp}{self}\PY{o}{.}\PY{n}{model\PYZus{}fit}\PY{o}{.}\PY{n}{summary}\PY{p}{(}\PY{p}{)}\PY{p}{)}
         
             \PY{k}{def} \PY{n+nf}{plot\PYZus{}residuals}\PY{p}{(}\PY{n+nb+bp}{self}\PY{p}{,} \PY{n}{lags}\PY{o}{=}\PY{l+m+mi}{20}\PY{p}{)}\PY{p}{:}        
                 \PY{c+c1}{\PYZsh{} plot residual errors}
                 
                 \PY{n+nb+bp}{self}\PY{o}{.}\PY{n}{residuals} \PY{o}{=} \PY{n}{pd}\PY{o}{.}\PY{n}{DataFrame}\PY{p}{(}\PY{n+nb+bp}{self}\PY{o}{.}\PY{n}{model\PYZus{}fit}\PY{o}{.}\PY{n}{resid}\PY{p}{)}
                 \PY{n+nb+bp}{self}\PY{o}{.}\PY{n}{residuals}\PY{o}{.}\PY{n}{plot}\PY{p}{(}\PY{p}{)}
                 \PY{n}{plt}\PY{o}{.}\PY{n}{grid}\PY{p}{(}\PY{p}{)}
                 \PY{n}{plt}\PY{o}{.}\PY{n}{title}\PY{p}{(}\PY{l+s+s1}{\PYZsq{}}\PY{l+s+s1}{fitted residual of Signal}\PY{l+s+s1}{\PYZsq{}}\PY{p}{)}
                 \PY{n}{plt}\PY{o}{.}\PY{n}{show}\PY{p}{(}\PY{p}{)}
                 \PY{n+nb+bp}{self}\PY{o}{.}\PY{n}{residuals}\PY{o}{.}\PY{n}{plot}\PY{p}{(}\PY{n}{kind}\PY{o}{=}\PY{l+s+s1}{\PYZsq{}}\PY{l+s+s1}{kde}\PY{l+s+s1}{\PYZsq{}}\PY{p}{)}
                 \PY{n}{plt}\PY{o}{.}\PY{n}{grid}\PY{p}{(}\PY{p}{)}
                 \PY{n}{plt}\PY{o}{.}\PY{n}{show}\PY{p}{(}\PY{p}{)}
                 \PY{n}{plot\PYZus{}acf}\PY{p}{(}\PY{n+nb+bp}{self}\PY{o}{.}\PY{n}{residuals}\PY{o}{.}\PY{n}{dropna}\PY{p}{(}\PY{p}{)}\PY{p}{,}\PY{n}{lags}\PY{o}{=}\PY{n}{lags}\PY{p}{)}
                 \PY{n}{plt}\PY{o}{.}\PY{n}{grid}\PY{p}{(}\PY{p}{)}
                 \PY{n}{plot\PYZus{}pacf}\PY{p}{(}\PY{n+nb+bp}{self}\PY{o}{.}\PY{n}{residuals}\PY{o}{.}\PY{n}{dropna}\PY{p}{(}\PY{p}{)}\PY{p}{,}\PY{n}{lags}\PY{o}{=}\PY{l+m+mi}{20}\PY{p}{)}
                 \PY{n}{plt}\PY{o}{.}\PY{n}{grid}\PY{p}{(}\PY{p}{)}
                 
             \PY{c+c1}{\PYZsh{} Accuracy metrics}
             \PY{k}{def} \PY{n+nf}{forecast\PYZus{}accuracy}\PY{p}{(}\PY{n+nb+bp}{self}\PY{p}{)}\PY{p}{:}
                 \PY{n}{mape} \PY{o}{=} \PY{n}{np}\PY{o}{.}\PY{n}{mean}\PY{p}{(}\PY{n}{np}\PY{o}{.}\PY{n}{abs}\PY{p}{(}\PY{n+nb+bp}{self}\PY{o}{.}\PY{n}{fc} \PY{o}{\PYZhy{}} \PY{n+nb+bp}{self}\PY{o}{.}\PY{n}{test}\PY{p}{)}\PY{o}{/}\PY{n}{np}\PY{o}{.}\PY{n}{abs}\PY{p}{(}\PY{n+nb+bp}{self}\PY{o}{.}\PY{n}{test}\PY{p}{)}\PY{p}{)}  \PY{c+c1}{\PYZsh{} MAPE}
                 \PY{c+c1}{\PYZsh{}me = np.mean(self.fc \PYZhy{} self.test)             \PYZsh{} ME}
                 \PY{c+c1}{\PYZsh{}mae = np.mean(np.abs(self.fc \PYZhy{} self.test))    \PYZsh{} MAE}
                 \PY{c+c1}{\PYZsh{}mpe = np.mean((self.fc \PYZhy{} self.test)/self.test)   \PYZsh{} MPE}
                 \PY{c+c1}{\PYZsh{}rmse = np.mean((self.fc \PYZhy{} self.test)**2)**.5  \PYZsh{} RMSE}
                 \PY{n}{corr} \PY{o}{=} \PY{n}{np}\PY{o}{.}\PY{n}{corrcoef}\PY{p}{(}\PY{n+nb+bp}{self}\PY{o}{.}\PY{n}{fc}\PY{p}{,} \PY{n+nb+bp}{self}\PY{o}{.}\PY{n}{test}\PY{p}{)}\PY{p}{[}\PY{l+m+mi}{0}\PY{p}{,}\PY{l+m+mi}{1}\PY{p}{]}   \PY{c+c1}{\PYZsh{} corr}
                 \PY{n}{mins} \PY{o}{=} \PY{n}{np}\PY{o}{.}\PY{n}{amin}\PY{p}{(}\PY{n}{np}\PY{o}{.}\PY{n}{hstack}\PY{p}{(}\PY{p}{[}\PY{n+nb+bp}{self}\PY{o}{.}\PY{n}{fc}\PY{p}{[}\PY{p}{:}\PY{p}{,}\PY{k+kc}{None}\PY{p}{]}\PY{p}{,} 
                                       \PY{n+nb+bp}{self}\PY{o}{.}\PY{n}{test}\PY{p}{[}\PY{p}{:}\PY{p}{,}\PY{k+kc}{None}\PY{p}{]}\PY{p}{]}\PY{p}{)}\PY{p}{,} \PY{n}{axis}\PY{o}{=}\PY{l+m+mi}{1}\PY{p}{)}
                 \PY{n}{maxs} \PY{o}{=} \PY{n}{np}\PY{o}{.}\PY{n}{amax}\PY{p}{(}\PY{n}{np}\PY{o}{.}\PY{n}{hstack}\PY{p}{(}\PY{p}{[}\PY{n+nb+bp}{self}\PY{o}{.}\PY{n}{fc}\PY{p}{[}\PY{p}{:}\PY{p}{,}\PY{k+kc}{None}\PY{p}{]}\PY{p}{,} 
                                       \PY{n+nb+bp}{self}\PY{o}{.}\PY{n}{test}\PY{p}{[}\PY{p}{:}\PY{p}{,}\PY{k+kc}{None}\PY{p}{]}\PY{p}{]}\PY{p}{)}\PY{p}{,} \PY{n}{axis}\PY{o}{=}\PY{l+m+mi}{1}\PY{p}{)}
                 \PY{n}{minmax} \PY{o}{=} \PY{l+m+mi}{1} \PY{o}{\PYZhy{}} \PY{n}{np}\PY{o}{.}\PY{n}{mean}\PY{p}{(}\PY{n}{mins}\PY{o}{/}\PY{n}{maxs}\PY{p}{)}             \PY{c+c1}{\PYZsh{} minmax}
                 \PY{c+c1}{\PYZsh{}acf1 = acf(self.fc\PYZhy{}self.test)[1]                      \PYZsh{} ACF1}
                 \PY{c+c1}{\PYZsh{}return(\PYZob{}\PYZsq{}mape\PYZsq{}:mape, \PYZsq{}me\PYZsq{}:me, \PYZsq{}mae\PYZsq{}: mae, }
                         \PY{c+c1}{\PYZsh{}\PYZsq{}mpe\PYZsq{}: mpe, \PYZsq{}rmse\PYZsq{}:rmse, \PYZsq{}acf1\PYZsq{}:acf1, }
                         \PY{c+c1}{\PYZsh{}\PYZsq{}corr\PYZsq{}:corr, \PYZsq{}minmax\PYZsq{}:minmax\PYZcb{})}
                 \PY{k}{return}\PY{p}{(}\PY{p}{\PYZob{}}\PY{l+s+s1}{\PYZsq{}}\PY{l+s+s1}{mape}\PY{l+s+s1}{\PYZsq{}}\PY{p}{:}\PY{n}{mape}\PY{p}{,} \PY{l+s+s1}{\PYZsq{}}\PY{l+s+s1}{corr}\PY{l+s+s1}{\PYZsq{}}\PY{p}{:}\PY{n}{corr}\PY{p}{,} \PY{l+s+s1}{\PYZsq{}}\PY{l+s+s1}{minmax}\PY{l+s+s1}{\PYZsq{}}\PY{p}{:}\PY{n}{minmax}\PY{p}{\PYZcb{}}\PY{p}{)}
                 
                 
             \PY{k}{def} \PY{n+nf}{forecast}\PY{p}{(}\PY{n+nb+bp}{self}\PY{p}{,} \PY{n}{factor}\PY{p}{,} \PY{n}{test\PYZus{}no}\PY{p}{,} \PY{n}{alpha}\PY{o}{=}\PY{l+m+mf}{0.005}\PY{p}{)}\PY{p}{:}
                 
                 \PY{n}{train} \PY{o}{=} \PY{n}{df}\PY{o}{.}\PY{n}{ClosePrice}\PY{o}{.}\PY{n}{drop}\PY{p}{(}\PY{n}{df}\PY{o}{.}\PY{n}{tail}\PY{p}{(}\PY{n}{test\PYZus{}no}\PY{p}{)}\PY{o}{.}\PY{n}{index}\PY{p}{)}\PY{o}{/}\PY{n}{factor}
                 \PY{n+nb+bp}{self}\PY{o}{.}\PY{n}{test} \PY{o}{=} \PY{n}{df}\PY{o}{.}\PY{n}{ClosePrice}\PY{o}{.}\PY{n}{tail}\PY{p}{(}\PY{n}{test\PYZus{}no}\PY{p}{)}\PY{o}{/}\PY{n}{factor}
                 
                 \PY{n}{model} \PY{o}{=} \PY{n}{ARIMA}\PY{p}{(}\PY{n}{train}\PY{p}{,} \PY{n}{order}\PY{o}{=}\PY{n+nb+bp}{self}\PY{o}{.}\PY{n}{order}\PY{p}{)}  
                 \PY{n}{fitted} \PY{o}{=} \PY{n}{model}\PY{o}{.}\PY{n}{fit}\PY{p}{(}\PY{n}{disp}\PY{o}{=}\PY{l+m+mi}{0}\PY{p}{)}  
                 \PY{n+nb}{print}\PY{p}{(}\PY{n}{fitted}\PY{o}{.}\PY{n}{summary}\PY{p}{(}\PY{p}{)}\PY{p}{)}
                 
                 \PY{c+c1}{\PYZsh{} Forecast}
                 \PY{n+nb+bp}{self}\PY{o}{.}\PY{n}{fc}\PY{p}{,} \PY{n}{se}\PY{p}{,} \PY{n}{conf} \PY{o}{=} \PY{n}{fitted}\PY{o}{.}\PY{n}{forecast}\PY{p}{(}\PY{n}{test\PYZus{}no}\PY{p}{,} \PY{n}{alpha}\PY{o}{=}\PY{n}{alpha}\PY{p}{)}  \PY{c+c1}{\PYZsh{} 95\PYZpc{} conf}
                 
                 \PY{c+c1}{\PYZsh{} Make as pandas series}
                 \PY{n}{fc\PYZus{}series} \PY{o}{=} \PY{n}{pd}\PY{o}{.}\PY{n}{Series}\PY{p}{(}\PY{n+nb+bp}{self}\PY{o}{.}\PY{n}{fc}\PY{p}{,} \PY{n}{index}\PY{o}{=}\PY{n+nb+bp}{self}\PY{o}{.}\PY{n}{test}\PY{o}{.}\PY{n}{index}\PY{p}{)}
                 \PY{n}{lower\PYZus{}series} \PY{o}{=} \PY{n}{pd}\PY{o}{.}\PY{n}{Series}\PY{p}{(}\PY{n}{conf}\PY{p}{[}\PY{p}{:}\PY{p}{,} \PY{l+m+mi}{0}\PY{p}{]}\PY{p}{,} \PY{n}{index}\PY{o}{=}\PY{n+nb+bp}{self}\PY{o}{.}\PY{n}{test}\PY{o}{.}\PY{n}{index}\PY{p}{)}
                 \PY{n}{upper\PYZus{}series} \PY{o}{=} \PY{n}{pd}\PY{o}{.}\PY{n}{Series}\PY{p}{(}\PY{n}{conf}\PY{p}{[}\PY{p}{:}\PY{p}{,} \PY{l+m+mi}{1}\PY{p}{]}\PY{p}{,} \PY{n}{index}\PY{o}{=}\PY{n+nb+bp}{self}\PY{o}{.}\PY{n}{test}\PY{o}{.}\PY{n}{index}\PY{p}{)}
                 
                 \PY{c+c1}{\PYZsh{} Plot}
                 \PY{n}{plt}\PY{o}{.}\PY{n}{figure}\PY{p}{(}\PY{n}{figsize}\PY{o}{=}\PY{p}{(}\PY{l+m+mi}{12}\PY{p}{,}\PY{l+m+mi}{5}\PY{p}{)}\PY{p}{,} \PY{n}{dpi}\PY{o}{=}\PY{l+m+mi}{100}\PY{p}{)}
                 \PY{n}{plt}\PY{o}{.}\PY{n}{plot}\PY{p}{(}\PY{n}{train}\PY{p}{,} \PY{n}{label}\PY{o}{=}\PY{l+s+s1}{\PYZsq{}}\PY{l+s+s1}{training}\PY{l+s+s1}{\PYZsq{}}\PY{p}{)}
                 \PY{n}{plt}\PY{o}{.}\PY{n}{plot}\PY{p}{(}\PY{n+nb+bp}{self}\PY{o}{.}\PY{n}{test}\PY{p}{,} \PY{n}{label}\PY{o}{=}\PY{l+s+s1}{\PYZsq{}}\PY{l+s+s1}{ClosePrice}\PY{l+s+s1}{\PYZsq{}}\PY{p}{)}
                 \PY{n}{plt}\PY{o}{.}\PY{n}{plot}\PY{p}{(}\PY{n}{fc\PYZus{}series}\PY{p}{,} \PY{n}{label}\PY{o}{=}\PY{l+s+s1}{\PYZsq{}}\PY{l+s+s1}{Forecast from Signal}\PY{l+s+s1}{\PYZsq{}}\PY{p}{)}
                 \PY{n}{plt}\PY{o}{.}\PY{n}{fill\PYZus{}between}\PY{p}{(}\PY{n}{lower\PYZus{}series}\PY{o}{.}\PY{n}{index}\PY{p}{,} \PY{n}{lower\PYZus{}series}\PY{p}{,} \PY{n}{upper\PYZus{}series}\PY{p}{,} 
                          \PY{n}{color}\PY{o}{=}\PY{l+s+s1}{\PYZsq{}}\PY{l+s+s1}{k}\PY{l+s+s1}{\PYZsq{}}\PY{p}{,} \PY{n}{alpha}\PY{o}{=}\PY{o}{.}\PY{l+m+mi}{15}\PY{p}{)}
                 \PY{n}{plt}\PY{o}{.}\PY{n}{title}\PY{p}{(}\PY{l+s+s1}{\PYZsq{}}\PY{l+s+s1}{Forecast vs ClosePrice}\PY{l+s+s1}{\PYZsq{}}\PY{p}{)}
                 \PY{n}{plt}\PY{o}{.}\PY{n}{legend}\PY{p}{(}\PY{n}{loc}\PY{o}{=}\PY{l+s+s1}{\PYZsq{}}\PY{l+s+s1}{upper left}\PY{l+s+s1}{\PYZsq{}}\PY{p}{,} \PY{n}{fontsize}\PY{o}{=}\PY{l+m+mi}{8}\PY{p}{)}
                 \PY{n}{plt}\PY{o}{.}\PY{n}{show}\PY{p}{(}\PY{p}{)}
                 
\end{Verbatim}


    \begin{Verbatim}[commandchars=\\\{\}]
{\color{incolor}In [{\color{incolor}29}]:} \PY{n}{ar111} \PY{o}{=} \PY{n}{Arima\PYZus{}model}\PY{p}{(}\PY{n}{ts}\PY{o}{=}\PY{n}{df}\PY{p}{[}\PY{l+s+s1}{\PYZsq{}}\PY{l+s+s1}{Signal}\PY{l+s+s1}{\PYZsq{}}\PY{p}{]}\PY{p}{,}\PY{n}{order}\PY{o}{=}\PY{p}{(}\PY{l+m+mi}{1}\PY{p}{,}\PY{l+m+mi}{1}\PY{p}{,}\PY{l+m+mi}{1}\PY{p}{)}\PY{p}{)}
\end{Verbatim}


    \begin{Verbatim}[commandchars=\\\{\}]
                             ARIMA Model Results                              
==============================================================================
Dep. Variable:               D.Signal   No. Observations:                  668
Model:                 ARIMA(1, 1, 1)   Log Likelihood                1393.087
Method:                       css-mle   S.D. of innovations              0.030
Date:                Mon, 20 Apr 2020   AIC                          -2778.175
Time:                        08:50:01   BIC                          -2760.158
Sample:                    01-04-2012   HQIC                         -2771.195
                         - 08-29-2014                                         
==================================================================================
                     coef    std err          z      P>|z|      [0.025      0.975]
----------------------------------------------------------------------------------
const              0.0026      0.000     18.386      0.000       0.002       0.003
ar.L1.D.Signal     0.9627      0.011     87.747      0.000       0.941       0.984
ma.L1.D.Signal    -1.0000      0.008   -120.710      0.000      -1.016      -0.984
                                    Roots                                    
=============================================================================
                  Real          Imaginary           Modulus         Frequency
-----------------------------------------------------------------------------
AR.1            1.0388           +0.0000j            1.0388            0.0000
MA.1            1.0000           +0.0000j            1.0000            0.0000
-----------------------------------------------------------------------------

    \end{Verbatim}

    The autoregressive term has a p-value that is less than the significance
level of 0.05. So I can conclude that the coefficient for the
autoregressive term is statistically significant.

Then let's check the residules.

    \begin{Verbatim}[commandchars=\\\{\}]
{\color{incolor}In [{\color{incolor}30}]:} \PY{n}{ar111}\PY{o}{.}\PY{n}{plot\PYZus{}residuals}\PY{p}{(}\PY{p}{)}
\end{Verbatim}


    \begin{center}
    \adjustimage{max size={0.9\linewidth}{0.9\paperheight}}{output_66_0.png}
    \end{center}
    { \hspace*{\fill} \\}
    
    \begin{center}
    \adjustimage{max size={0.9\linewidth}{0.9\paperheight}}{output_66_1.png}
    \end{center}
    { \hspace*{\fill} \\}
    
    \begin{center}
    \adjustimage{max size={0.9\linewidth}{0.9\paperheight}}{output_66_2.png}
    \end{center}
    { \hspace*{\fill} \\}
    
    \begin{center}
    \adjustimage{max size={0.9\linewidth}{0.9\paperheight}}{output_66_3.png}
    \end{center}
    { \hspace*{\fill} \\}
    
    From the figures above, we can see that the residules follow the normal
distribution. And no significant correlation are present, we can
conclude that the residuals are independent. Overall, it seems to be a
good fit. Let's forecast.

    We can also perform Ljung Box Test on the residues and the square values
of residues.

    \begin{Verbatim}[commandchars=\\\{\}]
{\color{incolor}In [{\color{incolor}31}]:} \PY{n+nb}{print}\PY{p}{(}\PY{n}{sm}\PY{o}{.}\PY{n}{stats}\PY{o}{.}\PY{n}{acorr\PYZus{}ljungbox}\PY{p}{(}\PY{n}{ar111}\PY{o}{.}\PY{n}{model\PYZus{}fit}\PY{o}{.}\PY{n}{resid}\PY{p}{,} \PY{n}{lags}\PY{o}{=}\PY{p}{[}\PY{l+m+mi}{1}\PY{p}{]}\PY{p}{,} \PY{n}{return\PYZus{}df}\PY{o}{=}\PY{k+kc}{True}\PY{p}{)}\PY{p}{)}
\end{Verbatim}


    \begin{Verbatim}[commandchars=\\\{\}]
   lb\_pvalue   lb\_stat
1   0.183195  1.771512

    \end{Verbatim}

    \begin{Verbatim}[commandchars=\\\{\}]
{\color{incolor}In [{\color{incolor}32}]:} \PY{n+nb}{print}\PY{p}{(}\PY{n}{sm}\PY{o}{.}\PY{n}{stats}\PY{o}{.}\PY{n}{acorr\PYZus{}ljungbox}\PY{p}{(}\PY{n}{ar111}\PY{o}{.}\PY{n}{model\PYZus{}fit}\PY{o}{.}\PY{n}{resid}\PY{o}{*}\PY{o}{*}\PY{l+m+mi}{2}\PY{p}{,} \PY{n}{lags}\PY{o}{=}\PY{p}{[}\PY{l+m+mi}{1}\PY{p}{]}\PY{p}{,} \PY{n}{return\PYZus{}df}\PY{o}{=}\PY{k+kc}{True}\PY{p}{)}\PY{p}{)}
\end{Verbatim}


    \begin{Verbatim}[commandchars=\\\{\}]
   lb\_pvalue   lb\_stat
1   0.749881  0.101631

    \end{Verbatim}

    \begin{Verbatim}[commandchars=\\\{\}]
{\color{incolor}In [{\color{incolor}33}]:} \PY{n}{ar111}\PY{o}{.}\PY{n}{model\PYZus{}fit}\PY{o}{.}\PY{n}{plot\PYZus{}predict}\PY{p}{(}\PY{n}{dynamic}\PY{o}{=}\PY{k+kc}{False}\PY{p}{)}
         \PY{n}{plt}\PY{o}{.}\PY{n}{grid}\PY{p}{(}\PY{p}{)}
         \PY{n}{plt}\PY{o}{.}\PY{n}{show}\PY{p}{(}\PY{p}{)}
\end{Verbatim}


    \begin{center}
    \adjustimage{max size={0.9\linewidth}{0.9\paperheight}}{output_71_0.png}
    \end{center}
    { \hspace*{\fill} \\}
    
    The above plot shows that the ARIMA(1,1,1) model can fit 'Signal' very
well. Next let's use it to forecast 'ClosePrice' of SP500 to see if
'Signal' could be predictive of future returns of the SP500 index.

Let's use the last 100 rows of dataframe df as test set and the rest of
the rows as training set. We use 'Signal' of the training set to train
the ARIMA(1,1,1) model. Then we test the model by forecasting the
'ClosePrice' in the test set. Note that we divided the 'ClosePrice' by
38 to make 'ClosePrice' and 'Signal' equal at the starting date of the
test set

    \begin{Verbatim}[commandchars=\\\{\}]
{\color{incolor}In [{\color{incolor}34}]:} \PY{n}{ar111}\PY{o}{.}\PY{n}{forecast}\PY{p}{(}\PY{n}{factor} \PY{o}{=} \PY{l+m+mi}{38}\PY{p}{,} \PY{n}{test\PYZus{}no}\PY{o}{=}\PY{l+m+mi}{100}\PY{p}{)}
\end{Verbatim}


    \begin{Verbatim}[commandchars=\\\{\}]
                             ARIMA Model Results                              
==============================================================================
Dep. Variable:           D.ClosePrice   No. Observations:                  568
Model:                 ARIMA(1, 1, 1)   Log Likelihood                1182.385
Method:                       css-mle   S.D. of innovations              0.030
Date:                Mon, 20 Apr 2020   AIC                          -2356.770
Time:                        08:50:04   BIC                          -2339.401
Sample:                    01-04-2012   HQIC                         -2349.992
                         - 04-08-2014                                         
======================================================================================
                         coef    std err          z      P>|z|      [0.025      0.975]
--------------------------------------------------------------------------------------
const                  0.0027      0.000     12.672      0.000       0.002       0.003
ar.L1.D.ClosePrice     0.9696      0.011     89.483      0.000       0.948       0.991
ma.L1.D.ClosePrice    -1.0000      0.007   -147.355      0.000      -1.013      -0.987
                                    Roots                                    
=============================================================================
                  Real          Imaginary           Modulus         Frequency
-----------------------------------------------------------------------------
AR.1            1.0313           +0.0000j            1.0313            0.0000
MA.1            1.0000           +0.0000j            1.0000            0.0000
-----------------------------------------------------------------------------

    \end{Verbatim}

    \begin{center}
    \adjustimage{max size={0.9\linewidth}{0.9\paperheight}}{output_73_1.png}
    \end{center}
    { \hspace*{\fill} \\}
    
    From the figure above, the ARIMA(1,1,1) model seems to give a
directionally correct forecast. And the actual observed 'ClosePrice' lie
within the 95\% confidence band.

    \begin{Verbatim}[commandchars=\\\{\}]
{\color{incolor}In [{\color{incolor}36}]:} \PY{n}{ar111}\PY{o}{.}\PY{n}{forecast\PYZus{}accuracy}\PY{p}{(}\PY{p}{)}
\end{Verbatim}


\begin{Verbatim}[commandchars=\\\{\}]
{\color{outcolor}Out[{\color{outcolor}36}]:} \{'mape': 0.017322513026562892,
          'corr': 0.8557742214558138,
          'minmax': 0.017311234266007558\}
\end{Verbatim}
            
    The MAPE, Correlation and Min-Max Error are used to evaluate the
forcast. Around 1.7\% MAPE implies the model is about 98.3\% accurate in
predicting the next 100 trading days. So I think this forcast is quite
good.

    \section{3 Conclusion.}\label{conclusion.}

    \begin{enumerate}
\def\labelenumi{(\arabic{enumi})}
\tightlist
\item
  From the data cleaning section. We find that there are 4 days which
  are illegal.
\end{enumerate}

\begin{itemize}
\tightlist
\item
  2013-12-25 is Christmas Day
\item
  2014-01-01 is New Year's Day
\item
  2014-02-08
\item
  2014-02-09 are weekends.
\end{itemize}

What's more, there are 6 outliers in the 'Signal' column, which are -
2013-11-05 - 2013-11-06 - 2013-03-26 - 2014-04-14 - 2014-04-15 -
2014-04-16.

Also, there are 3 outliers in the 'ClosePrice' column, which are -
2013-09-12 - 2013-09-13 - 2013-09-16.

In addition, there are 6 missing dates which are actually trading days
but are absent in the table. They are - 2013-01-14 - 2013-01-15 -
2013-01-16 - 2013-01-17 - 2014-01-06 - 2014-02-11

    \begin{enumerate}
\def\labelenumi{(\arabic{enumi})}
\setcounter{enumi}{1}
\tightlist
\item
  By training the ARIMA model using 'Signal', we can predict future
  values of 'ClosePrice'. So 'Signal'can be predictive of future returns
  of the SP500 index (use SPY as a proxy).
\end{enumerate}

However, we should tell the Portfolio Manager that 'ClosePrice' is
approximately 40 times bigger than 'Signal'. We should calculate this
factor today before we forecast the 'Closeprice' tomorrow.


    % Add a bibliography block to the postdoc
    
    
    
    \end{document}
